\documentclass[11pt,a4paper,english]{report}
\usepackage[english]{babel}

\usepackage{a4wide}
\usepackage{times}
\usepackage[english]{babel}
\usepackage[pdfstartview=FitH,a4paper=true,pdfsubject="The LuaCOM Manual",pdfauthor="Vinicius Almendra and Renato Cerqueira and Fabio Mascarenhas",pdftitle="The LuaCOM Manual",colorlinks=true,citecolor=black,linkcolor=black,pagecolor=black,urlcolor=black]{hyperref}

\special{!userdict begin /bop-hook{gsave 210 40 translate
65 rotate /Times-Roman findfont 216 scalefont setfont
0 0 moveto 0.97 setgray (DRAFT) show grestore}def }

\newcommand{\ttext}[1]{{\raggedright #1 \\}}
\newcommand{\ttitle}[1]{\begin{minipage}{\colA cm} \centering \bf #1 \end{minipage}}

\newcommand{\tlineB}[2]{\ttext{#1} & \ttext{#2} \\ \hline}
\newcommand{\tlinetitB}[2]{\ttitle{#1} & \ttitle{#2} \\ \hline \hline}

\newcommand{\ident}[1]{\mbox{\texttt{#1}}}
%\newcommand{\ident}[1]{\texttt{#1}}

% IMPROVE: use \xspace here to avoid trailing \
\newcommand{\techname}[1]{\mbox{#1}}

\newcommand{\lua}{\techname{Lua}}
\newcommand{\com}{\techname{COM}}

\newcommand{\activex}{\techname{ActiveX}}
\newcommand{\luacom}{\techname{LuaCOM}}

\newcommand{\idispatch}{\ident{IDispatch}}
\newcommand{\dispinterface}{\ident{dispinterface}}
\newcommand{\dispinterfaces}{\ident{dispinterfaces}}

\newcommand{\registered}{$^{\textregistered}$}


\title{\luacom \ User Manual \\
{\small (Version 1.4)}}

\author{Vinicius Almendra \and Renato Cerqueira \and Fabio Mascarenhas}

\begin{document}

\maketitle

\tableofcontents

\chapter{Introduction}
  \luacom\ is an add-on library to the Lua language that allows Lua
  programs to use and implement objects that follow Microsoft's
  \emph{Component Object Model} (\com) specification {\bf and} use the
  Automation technology for property access
  and method calls.

\section{Features}

  Currently, the \luacom\ library supports the following features:

  \begin{itemize}

    \item dynamic instantiation of \com\ objects registered in the
    system registry, via the \verb+CreateObject+ method;

    \item dynamic access to running \com\ objects via
    \verb+GetObject+;

    \item \com\ method calls as normal Lua function calls and property
    accesses as normal table field accesses;

    \item ability to read type libraries and to generate HTML
    documentation on-the-fly for COM objects;

    \item use of COM objects without type information;

    \item type conversion between OLE Automation types and Lua types;

    \item object disposal using Lua garbage collection mechanism;

    \item implementation of \com\ interfaces and objects using Lua
    tables;

    \item implementation of OLE controls using Lua tables (needs a Lua
    GUI toolkit that can create in-place windows, like IUP);

    \item use of \com\ connection point mechanism for bidirectional
    communication and event handling;

    \item fully compatible with Lua 5.1;

    \item log mechanism to ease the debugging of applications.

  \end{itemize}

\section{Who Should Read What (or About the Manual)}
This manual is mostly a reference manual. Here we document the
behavior of \luacom\ in a variety of situations, some implementation
decisions that affect the end-user of the library and its
limitations. When facing some strange behavior in an application built
using \luacom, the first step is to read all the chapter
\ref{chap:release}, where the majority of possible problems are
documented. There can be found references to other sections of the
manual, where more detailed information is provided.

\paragraph{Newbies} For those who are newcomers, we provide a tutorial
section (chapter \ref{cap:tutorial}) with a step-by-step guide to
start using \luacom. More help and samples can be found in \luacom's
home page and in the demos folder. Notice that VBScript code can be
easily converted to \lua\ with \luacom.

This manual does not provide information for developers who need
deeper technical information about \luacom\ or who are willing to
modify it for some reason. For this kind of information, please
contact the authors.

\paragraph{Knowledge required} This manual presumes some knowledge of
\com\ and Automation. We don't intend to explain in detail how these
technologies work or how they can be used to solve particular
problems. This information can be found easily in the web or in good
books.

\section*{Some information about samples} The sample codes shown in
this documentation are all for Lua 5.1. Samples for previous Lua versions
can be found in the documentation for the previous versions of \luacom.

\chapter{Tutorial}
\label{cap:tutorial}

\section{Using The \luacom\ library}

  \luacom\ is an add-on to the Lua language. To be used, either the binary
  library of \luacom\ must be linked with the host program, just like
  the Lua library and other add-ons, or you should load a \luacom\ dynamic
  library through Lua 5's module require mechanism.
  There are different versions of the \luacom\ binary for the different
  versions of the Lua library, so pay attention to link the right one.

  If you are linking \luacom\ to your program, the next step is to modify
  the source code of the host program to
  call \luacom's \emph{and} COM initialization and termination
  functions, which are part of the C/C++ API. To do so, include the
  \luacom 's header --- \verb+luacom.h+ --- and call these functions
  in the proper order: \luacom\ must be initialized after both COM and Lua;
  \luacom\ must be terminated before Lua, and Lua must be terminated
  before COM
  \footnote{Notice that COM must be initialized in each thread that
  will use it. To use \luacom\ in this situation, it's not safe to
  share a single Lua state among several threads; one should create a
  new Lua state for each thread and then initialize \luacom\ with this
  state.}. Here is an example of a simple \ident{C} host program
  program using \luacom.

{\small
\begin{verbatim}
   /*
    * Sample C program using luacom
    */
   #include <stdio.h>
   #include <ole2.h> // needed for CoInitialize and CoUninitialize
   #include <lua.h>
   #include <luacom.h>

   int main (int argc, char *argv[]) {
     /* COM initialization */
     CoInitialize(NULL);
     /* library initialization */
     lua_State *L = luaL_newstate();
     luacom_open(L);
     if(luaL_dofile("luacom_sample.lua") != 0) {
       puts("Error running sample!");
       exit(1);
     }
     luacom_close(L);
     lua_close(L);
     CoUninitialize(NULL);
     return 0;
   }
\end{verbatim}}
\noindent
Notice that it's necessary to initialize \com\ before
\verb+luacom_open+ and to terminate it only after the last
\verb+lua_close+, otherwise fatal errors may occur.

Using Lua to dynamically load \luacom\ is simpler. Just call
\verb+require "luacom"+ in your Lua script, and make sure the file
\verb+luacom.dll+ is in your \verb+LUA_CPATH+ environment variable and
that Lua's DLLs (e.g. \verb+lua5.1.dll+) are in your \verb+PATH+.

%OLD 1.3: \verb+luacom.lua+ is in your \verb+LUA_PATH+ environment
% variable, and the Lua and \luacom\ DLLs (\verb+lua-5.0.dll+,
% \verb+lualib-5.0.dll+ and
% \verb+luacom-lua5-1.3.dll+, respectively) are in your \verb+PATH+.
% Then run your script with the Lua standalone interpreter.

\section{Locating \com\ Objects}

The first step to use a COM object is to find it. COM objects are
registered in the system registry and are associated with an unique
Class Identifier, known as CLSID. A CLSID may also be associated with
a string known as Programmatic Identifier or ProgID. This last one is
the easiest way to reference a COM object. For example, the ProgID for
Microsoft\registered\ Word is ``\ident{Word.Application}''.

If one does not know in advance what is the CLSID or the ProgID of the
object of interest, then it's possible to use tools like OleView to
find the object, though the best place to find it is in the object's
documentation.


\section{Creating Objects}

With the ProgID or the CLSID of an object, it's now possible to create
a new instance of it or to get a running instance. To do so, the
easiest way is to use the method CreateObject of the Lua API:

\begin{verbatim}
  word = luacom.CreateObject("Word.Application")
  assert(word)
  word.Visible = true
\end{verbatim}

If there is an already running instance of the object you want,
\ident{GetObject} must be used to use it. The following code
illustrates this:

\begin{verbatim}
  -- If there is an instance of Word(r) running,
  -- it will end it
  word = luacom.GetObject("Word.Application")
  if word then
    word:Quit()
    word = nil
  end
\end{verbatim}

\section{Getting Help about an Object}

To use a COM object, the first thing one must know is its
\emph{interface}---that is, its set of methods and properties. This
information normally is available in the documentation of the object,
but sometimes one do not have access to this documentation. \luacom\
can offer some help if the object has type information. If the object
has an associated help file, \luacom\ can launch it using the method
\ident{ShowHelp}:

\begin{verbatim}
  word = luacom.CreateObject("Word.Application")
  assert(word)
  luacom.ShowHelp(word)
\end{verbatim}

If the object has an associated type library, \luacom\ can generate
and display an HTML file describing it. This information can also be
read using other type library browsers, as OleView.

The method \ident{DumpTypeInfo} can be used in console applications to
list the methods and properties of the interface. It does not give
much information, but can be helpful when playing with an object.


\section{Methods and Properties}

After creating an object, the next step is to use it. This is
primarily done through \emph{method calls} and \emph{property
accesses}. To call a method of the object, do it as if the method was
a function stored in a Lua table whose key is the method name:

\begin{verbatim}
-- Here we call the method 'Show' of the COM object
myobj:Show()
-- A method with a return value
result = myobj:CheckState()
-- A method with parameters
file = myobj:LoadFile("test.xyz", 1)
-- A method with output values
x, y = myobj:UpdatePosition(x, y)
\end{verbatim}

To read or write simple properties, one must simply use them as if
they were normal table fields.

\begin{verbatim}
-- Reading properties
value1 = obj1.Value
value1 = obj2.Value
-- writing a property
obj3.Value = value1 + value2
\end{verbatim}

Automation includes support to \emph{parametrized properties}. These
can be accessed (or written) using accessor functions.

\begin{verbatim}
value = obj:getMatrixValue(1,1)
value = value*0,125
obj:setMatrixValue(1, 1, value)
\end{verbatim}



\section{Releasing Objects}
Objects are automatically released using Lua's garbage collection
mechanism, that is, when there are no references to them in
Lua. However, some objects may demand an explicit termination method
call, like ``Quit''.

\begin{verbatim}
obj = luacom.CreateObject("MyApp.MyObj")
-- Here we force an immediate release of the object
obj = nil
collectgarbage()
\end{verbatim}

Notice that if there is any references to the COM object alive in
\lua\, then the application (or library) that implements it will not
exit.



\chapter{\luacom\ Elements}

  \luacom\ is composed by the following elements:

  \begin{itemize}

    \item \luacom\ objects, which make COM objects available in Lua;

    \item \luacom\ API, a set of functions used to do a variety of
    tasks (library initialization, object creation, implementation of
    Automation interfaces in Lua, manipulation of connection points,
    etc.);

    \item Automation binding, which translates accesses on \luacom\
    objects to COM interface calls and COM accesses on an interface
    implemented in Lua to Lua function calls or table accesses;

    \item \luacom\ type conversion rules, which govern the type conversion
    between Lua and Automation types;

    \item \luacom\ parameter passing rules, which describe how \luacom\
    translate a \lua\ parameter list to a \com\ one and vice versa;

    \item other objects, like typelib, typeinfo, enumerator, etc.

  \end{itemize}

%%%
%   LuaCOM API
%%%

\section{\luacom\ API}

  The \luacom\ API is divided in two parts: the standard API and the
  extended API. The standard API comprises the core functionality
  needed to use \com\ objects. The extended API includes more advanced
  features to Lua API that simplify the development of applications
  using \luacom. This distinction has been made due to the possible
  unbounded growth of features, which could end up cluttering the
  library and making it bigger and bigger and more and more difficult
  to use. For now, the extended API is entirely implemented in Lua
  and can be easily removed without trouble.

  The standard API is further divided in two classes: the Lua API and
  the C/C++ API. The C/C++ API is used primarily for initialization of
  the library and for low-level construction of \luacom\ objects. It
  is declared in the header file \ident{luacom.h}. The Lua API
  permits Lua programs to access all the functionality of \luacom. It
  is implemented as a set of functions inside a global table named
  \verb+luacom+; hereafter, these functions will be called LuaCOM
  \emph{methods}. This table is created and populated when the C/C++ API function
  \ident{luacom\_open} is called. Below there is summary of
  the \luacom\ API. Detailed information on these methods is available
  in chapter \ref{chap:reference}.

  \newcommand{\colA}{6}

%%% Lua API table

  \begin{center}
  \begin{tabular}{|c|p{7cm}|}

  \multicolumn{2}{c}{\large Standard Lua API} \\
  \hline

  \tlinetitB{Method}{Description}

  \verb+CreateObject+
    & \ttext{Creates a \luacom\ object.} \\ \hline

  \verb+NewObject+
    & \ttext{Creates a \luacom\ object implemented in Lua.} \\ \hline

  \verb+NewControl+
    & \ttext{Creates a \luacom\ OLE control implemented in Lua.} \\ \hline

  \verb+GetObject+
    & \ttext{Creates a \luacom\ object associated with an
    instance of an already running COM object.} \\ \hline

  \verb+ExposeObject+ & \ttext{Exposes a \luacom\ object or OLE control, so
    that other applications can get a reference to it.} \\ \hline

  \verb+RevokeObject+
    & \ttext{Undoes the operation of \texttt{ExposeObject}.} \\ \hline

  \verb+RegisterObject+
    & \ttext{Fills in the registry entries necessary for exposing a
    \com\ object or OLE control.} \\ \hline

  \verb+UnRegisterObject+
    & \ttext{Removes the registry entries necessary for exposing a
    \com\ object or OLE control.} \\ \hline

  \verb+Connect+
    & \ttext{Creates a connection point between an object and a Lua table.} \\ \hline

  \verb+ImplInterface+ & \ttext{Implements an IDispatch
    interface using a Lua table.} \\ \hline

  \verb+ImplInterfaceFromTypelib+ & \ttext{Implements an
    IDispatch interface described in a Type Library using a Lua table.}
    \\ \hline

  \verb+addConnection+
    & \ttext{Connects two \luacom\ objects.} \\ \hline

  \verb+releaseConnection+
    & \ttext{Disconnects a \luacom\ object from its
    connection point.} \\ \hline

  \verb+isMember+
    & \ttext{Checks whether a name correspond to a method
    or a property of an \luacom\ object.} \\ \hline

  \verb+ProgIDfromCLSID+
    & \ttext{Gets the ProgID associated with a CLSID.} \\ \hline

  \verb+CLSIDfromProgID+
    & \ttext{Gets the CLSID associated with a ProgID.} \\ \hline

  \verb+GetIUnknown+
    & \ttext{Returns an \ident{IUnknown} interface to a \luacom\
    object as a full userdata.} \\ \hline

  \verb+DumpTypeInfo+ & \ttext{Dumps to the console the type
    information of the specified \luacom\ object. This method should
    be used only for debugging purposes.} \\ \hline

  \end{tabular}
  \end{center}

  \begin{center}
  \begin{tabular}{|c|p{7cm}|}

  \multicolumn{2}{c}{\large Standard Lua API (continued)} \\
  \hline

  \tlinetitB{Method}{Description}

  \verb+GetCurrentDirectory+ & \ttext{Returns the current directory.} \\ \hline

  \verb+CreateLuaCOM+ & \ttext{Transforms an IUnknown full userdata
  into a LuaCOM object.} \\ \hline

  \verb+ImportIUnknown+ & \ttext{Converts a light userdata (pointer)
  to an IUnknown full userdata.} \\ \hline

  \verb+DetectAutomation+ & \ttext{Used to implement \com\ servers.
  Looks in the command-line for /Register or /UnRegister /Automation (not case-sensitive)
  and calls user-defined functions to register, unregister, or expose objects, entering
  a message loop in the latter case. If there is no command-line then assume it is being
  run in-process, calls the expose function and returns.
  } \\ \hline

  \end{tabular}
  \end{center}

%% Extended API table

  \begin{center}
  \begin{tabular}{|c|p{7cm}|}

  \multicolumn{2}{c}{\large Extended Lua API} \\
  \hline

  \tlinetitB{Method}{Description}

  \verb+CreateLocalObject+
    & \ttext{Creates a \luacom\ object as an out-of-process server.} \\ \hline

  \verb+CreateInprocObject+
    & \ttext{Creates a \luacom\ object as an in-process server.} \\ \hline

  \verb+ExportConstants+ & \ttext{Exports all the constants of a type
  library (standalone or bound to a LuaCOM object) to the global
  environment (or optionally to a table).} \\ \hline

  \verb+DumpTypeLib+
    & \ttext{Creates an HTML file describing a type library.} \\ \hline

  \verb+GetType+
    & \ttext{Returns a string describing the type of object, in the
    case its an object belonging to the \luacom\ library.} \\ \hline

  \verb+ViewTypeLib+
    & \ttext{Runs DumpTypeLib and shows the created file using
    Internet Explorer\registered.} \\ \hline

  \verb+pairs+ & \ttext{Does the same as pairs for COM Enumerators.}
  \\ \hline

  \verb+FillTypeLib+ & \ttext{Creates a table describing a type
    library.}
    \\ \hline

  \verb+FillTypeInfo+
    & \ttext{Creates a table describing a type info.} \\ \hline

  \end{tabular}
  \end{center}




%%% C/C++ API table

  \begin{center}
  \begin{tabular}{|c|p{6cm}|}

  \multicolumn{2}{c}{\large Standard C/C++ API} \\
  \hline

  \tlinetitB{Function}{Description}

  \verb+luacom_open+ & \ttext{Initializes the \luacom\ library in a
    \lua\ state. It must be called before any use of \luacom\
    features.} \\ \hline

  \verb+luacom_close+
    & \ttext{\luacom's termination function.} \\ \hline

  \verb+luacom_detectAutomation+
    & \ttext{This function is a helper to create COM servers. It looks
    in the command line for the switches ``/Automation'' and
    ``/Register'' and call some user-defined Lua functions accordingly.} \\ \hline


  \verb+luacom_IDispatch2LuaCOM+
    & \ttext{Takes an IDispatch interface and creates a \luacom\
    object to expose it, pushing the object on the Lua stack.} \\ \hline

  \end{tabular}
  \end{center}



\section{\luacom\ objects}
  \luacom\ deals with \emph{\luacom\ objects}, which are no more than
  a Lua table with the \luacom\ metatable and a reference to the
  \luacom\ C++ object; this one is, in turn, a proxy for the COM
  object: it holds an \idispatch\ pointer to the object and translates
  Lua accesses to Automation calls and property accesses. Here is a
  sample where a \luacom\ object is used:

\begin{verbatim}
-- Instantiate a Microsoft(R) Calendar Object
calendar = luacom.CreateObject("MSCAL.Calendar")
-- Error check
if calendar == nil then
  print("Error creating object")
  exit(1)
end
-- Method call
calendar:AboutBox()
-- Property Get
current_day = calendar.Day
-- Property Put
calendar.Month = calendar.Month + 1
print(current_day)
print(calendar.Month)
\end{verbatim}

Every time \luacom\ needs to convert an \idispatch\ pointer to Lua it
creates a \luacom\ object. There are two situations where this
happens:

\begin{itemize}
\item when calling \luacom\ API functions that return COM objects
(\ident{CreateObject}, \ident{GetObject}, \ident{NewObject},
\ident{Connect}, etc.) and
\item when receiving return values from COM, where some of the values
are \idispatch\ pointers.
\end{itemize}

Here follows a sample of these situations:

\begin{verbatim}
-- First, we get a luacom object using LuaCOM API
excel = luacom.CreateObject("Excel.Application")
assert(luacom.GetType(excel) == "LuaCOM")
-- now we get one from a method call
sheets = excel.Sheets
assert(luacom.GetType(sheets) == "LuaCOM")
\end{verbatim}

A \luacom\ object may be passed as a parameter to method calls on
other \luacom\ objects, if these methods expect an argument of type
\dispinterface. Here is a sample to illustrate this situation:

\begin{verbatim}
-- Gets a running instance of Excel
excel = luacom.GetObject("Excel.Application")
-- Gets the set of worksheets
sheets = excel.Worksheets
-- gets the first two sheets
sheet1 = sheets:Item(1)
sheet2 = sheets:Item(2)
-- Exchange them (here we pass the second sheet as a parameter
-- to a method)
sheet1:Move(nil, sheet2)
\end{verbatim}


There are two kinds of \luacom\ objects: \emph{typed} and
\emph{generic} ones. The typed ones are those whose COM object has
type information. The generic ones are those whose COM object does not
supply any type information. This distinction is important in some
situations.



\subsection{Object Disposal}
\label{sec:object-disposal}

\luacom\ objects are released through Lua's garbage collection
mechanism, so there isn't any explicit API method to destroy them.

\paragraph{Caution} \luacom\ only tracks references to COM objects. It
does not work with the concepts of ``application'', ``component'',
``process'', etc. It does not know even which objects are part of the
same component or application. This has some consequences on the
object disposal:

\begin{itemize}
\item a component may only consider as ``finished'' its relationship
with \luacom\ when all references to its objects are released, not
only the one created with \ident{CreateObject};
\item some components have a ``Quit'' method. This may close the
component's interface, but it could remain running if there are any
references to it. Nevertheless, these references cannot be reliably
used after the ``Quit'' method has been called. To release the
component, one must assign \ident{nil} to all references to the
component (and its sub-objects) and then call \ident{collectgarbage}.
\end{itemize}




\section{Automation binding}

  The Automation binding is responsible for translating the table
  accesses to the \luacom\ object into COM interface calls. Besides
  that, it also provides a mechanism for implementing \dispinterfaces\
  using ordinary Lua tables.




\subsection{Implementing \dispinterfaces\ in \lua}

The Automation binding has a \ident{C++} class called
\ident{tLuaDispatch} that implements a generic \idispatch\
interface. The implementation of this class translates the method
calls and property accesses done on the objects of this class to \lua\
calls and table accesses. So, one may implement a \dispinterface\
entirely in \lua, provided it has a type library describing it. This
type library may be a stand-alone one (referenced by its location on
the file system) or may be associated with some registered
component. In this case, it may be referenced by the \ident{ProgID} of
the component.

The \ident{C++} objects of this class can be used in any place where
an \idispatch\ or \ident{IUnknown} interface is expected. \luacom\
takes care of these conversion. Here follows a sample implementation of a
\dispinterface\ in Lua.

\begin{verbatim}
-- Creates and fills the Lua table that will implement the
-- COM interface
events_table = {}
function events_table:AfterUpdate()
  print("AfterUpdate called!")
end
-- Here we implement the interface DCalendarEvents, which is part
-- of the Microsoft(R) Calendar object, whose ProgID is MSCAL.Calendar
events_obj = luacom.ImplInterface(
  events_table,
  "MSCAL.Calendar",
  "DCalendarEvents")
-- Checks for errors
--
if events_obj == nil then
  print("Implementation failed")
  exit(1)
end
-- Tests the interface: this must generate a call to the events:AfterUpdate
-- defined above
--
events_obj:AfterUpdate()
\end{verbatim}

  If the interface to be implemented is described in a stand-alone
  type library, the method \verb+ImplInterfaceFromTypelib+
  must be used instead:



\begin{verbatim}
-- Creates and fills the Lua table that will implement the
-- Automation interface
hello_table = {}
function hello:Hello()
  print("Hello World!")
end
-- Here we implement the interface IHello
--
hello_obj = luacom.ImplInterfaceFromTypelib(hello_table, "hello.tlb", "IHello")
-- Checks for errors
--
if hello_obj == nil then
  print("Implementation failed")
  os.exit(1)
end
-- Tests the interface
--
hello_obj:Hello()
\end{verbatim}

  Both methods return a \luacom\ object, whose corresponding
  \idispatch\ interface is implemented by the supplied table. This
  \luacom\ object can be passed as an argument to COM methods who
  expect a \dispinterface\ or to \luacom\ API methods (like
  \verb+addConnection+).

  One can also use the \verb+NewObject+ method, which is best suited
  to the situation where one needs to create a complete component in
  \lua\ and wants to export it so that it can be accessed through COM
  by any running application.



\subsection{Using Methods and Properties}
\label{sec:using-meth-prop}

The \dispinterfaces\ have two ``types'' of members: properties and
methods. \luacom\ deals with both.

Method accesses are done in the same way as calling \lua\ functions
stored in a table and having a ``self'' parameter:

\begin{verbatim}
obj = luacom.CreateObject("TEST.Test")
if obj == nil then
  exit(1)
end
-- method call
a = obj:Teste(1,2)
-- another one
obj:Teste2(a+1)
\end{verbatim}

It's important to notice the need of using the colon -- ``:'' -- for
method calls. Although \luacom\ does not use the \ident{self}
parameter that \lua\ passes in this case, its presence is assumed ---
that is, \luacom\ always skips the first parameter in the case of
method calls; forgetting it may cause nasty bugs. Notice that this
rule doesn't apply when using the default method of a \luacom\ object
stored in a table or in a property of another \luacom\ object (see
section \ref{sec:default-methods} below).

Accessing properties is much like the same of accessing fields in
\lua\ tables:

\begin{verbatim}
obj = luacom.CreateObject("TEST.Test")
if obj == nil then
  exit(1)
end
-- property access
a = obj.TestData
-- property setting
obj.TestData = a + 1
\end{verbatim}


Properties may also be accessed as methods. This is mandatory when
dealing with parameterized properties --- that is, ones that accept (or
demand) parameters. A common example of this situation is the ``Item''
property of collections.

\begin{verbatim}
-- property access
a = obj:TestData()
-- Parametrized property access
b = obj:TestInfo(2)
-- Accessing collections
c = obj.Files:Item(2)
\end{verbatim}

Notice that the colon -- ``:'' -- must also be used in this situation.

When accessing properties with method calls, \luacom\ always
translates the method call to a read access (property get). To set the
value of a property using a method call, it's necessary append the
prefix ``set''\footnote{In a future version it might be allowed to
change the prefix.} to the property name and the new value must be
supplied as the last argument.

\begin{verbatim}
-- property access
a = obj:TestData()
-- Setting the property
b = obj:setTestInfo(2)
-- Setting a parametrized property
c = obj.Files:setItem(2, "test.txt")
\end{verbatim}

The prefix ``get'' may also be used, to clarify the code, although
it's not necessary, as the default behavior is to make a read access.

\begin{verbatim}
-- property access
a = obj:getTestData()
b = obj:getTestInfo(2)
c = obj.Files:getItem(2)
\end{verbatim}

\subsubsection{Extensible interfaces}
\label{sec:extens-interf}
\luacom\ allows the use of properties as simple Lua fields just for
objects that have type information. Nevertheless, some objects that
\emph{have} type information describing their interfaces implement
properties that are not described in the type library: these objects
implement \emph{extensible} interfaces. Those properties can only be
used with accessor functions, as shown in section
\ref{sec:gener-luac-objects}. An example of such behaviour is found in
WMI objects (Windows Management Instrumentation).

\subsubsection{Default methods}
\label{sec:default-methods}
A \dispinterface\ can have a default method or property --- that is, one
that is called when the client does not specify the method
name. \luacom\ calls the default method when the object itself is used
as a function.

\begin{verbatim}
  excel = luacom.CreateObject("Excel.Application")
  excel.Visible = true
  excel.Workbooks:Add()
  -- Here we call the default method
  -- notice we DID NOT use the colon, as
  -- the object used is Sheets, not excel
  sheet = excel.Sheets(1)
  print(sheet.Name)
  -- Here we also call the default method
  sheets = excel.Sheets
  sheet2 = sheets(2)
  print(sheet2.Name)
  -- Setting values
  excel.Sheets(1).Name = "MySheet1"
  excel:Quit()
\end{verbatim}

This can be very useful when dealing with collections, as commonly
they have a default \ident{Item} property.

WARNING: one must be careful not to put the colon when using default methods of
\luacom\ objects contained in table or in other \luacom\ objects (see
the sample above).




\subsubsection{Generic \luacom\ objects}
\label{sec:gener-luac-objects}
To read or write properties in generic \luacom\ objects, it's
necessary access them as method calls with the right prefix
(get/set). The simpler semantic of table field access does not work
here.

\begin{verbatim}
obj_typ = luacom.CreateObject("Some.TypedObject")
obj_untyp = luacom.CreateObject("Untyped.Object")
-- property read (get)
a = obj_typ.Value
b = obj_untyp:getValue()
-- property write (set)
obj.typ = a + 1
obj_untyp:setValue(b + 1)
\end{verbatim}


\subsubsection{Property Access in \lua}

When implementing a \com\
interface in \lua, \luacom\ also supports the concepts of properties and
indexed properties. \luacom\ translate property reads and writes to
table field accesses:

\begin{verbatim}
interface = {}
interface.Test = 1
interface.TestIndex = {2,3}
obj = luacom.ImplInterface(interface, "TEST.Test", "ITest")
-- must print "1"
print(obj.Test)
-- must print nil (if there is no member named Test2)
print(obj.Test2)
-- this writes the field Test
obj.Test = 1
-- Indexed property read. Must return 3 (remember that
-- indexed tables start at 1 in Lua)
i = obj:TestIndex(2)
-- Sets the indexed field
obj:setTestIndex(2,4)
-- Now must return 4
i = obj:TestIndex(2)
\end{verbatim}


\subsection{Connection Points: handling events}
\label{sec:conn-points:-handl}
  The \emph{connection points} are part of a standard \activex\
  mechanism whose primary objective is to allow the \activex\ object
  to notify its owner of any kind of events. The connection point
  works as an ``event sink'', where events and notifications go
  through.

  To establish a connection using \luacom, the owner of the \activex\ object
  must create a table to implement the connection interface,
  whose description is provided by the \activex\ object (this interface
  is called a \emph{source} interface) and then call the API method
  \verb+Connect+, passing as arguments the \luacom\ object
  for the \activex\ object and the implementation table. Doing this,
  \luacom\ will automatically find the default source interface, create
  a \luacom\ object implemented by the supplied table and then connect
  this object to the \activex\ object. Here follows a sample:


\begin{verbatim}
-- Creates the COM object
--
calendar = luacom.CreateObject("MSCAL.Calendar")
if calendar == nil then
  os.exit(1)
end
-- Creates implementation table
--
calendar_events = {}
function calendar_events:AfterUpdate()
  print("Calendar updated!")
end
-- Connects object and table
--
res, cookie = luacom.Connect(calendar, calendar_events)
if res == nil then
  exit(1)
end
-- This should trigger the AfterUpdate event
--
calendar:NextMonth()
\end{verbatim}

  The cookie returned by \verb+Connect+ identifies this connection, and can
  later be used to release the Connection. A COM object can have several event
  sinks connected to it simultaneously.

  It's also possible to separately create a \luacom\ object implementing the
  connection point source interface and then connect it to the object using
  \verb+addConnection+.

\begin{verbatim}
-- Instances the COM object
--
calendar = luacom.CreateObject("MSCAL.Calendar")
if calendar == nil then
  print("Error instantiating calendar")
  os.exit(1)
end
-- Creates implementation table
--
calendar_events = {}
function calendar_events:AfterUpdate()
  print("Calendar updated!")
end
-- Creates LuaCOM object implemented by calendar_events
--
event_handler = luacom.ImplInterface(calendar_events,
  "MSCAL.Calendar",
   "DCalendarEvents")
if event_handler == nil then
  print("Error implementing DCalendarEvents")
  exit(1)
end
-- Connects both objects
--
cookie = luacom.addConnection(calendar, event_handler)
-- This should trigger the AfterUpdate event
--
calendar:NextMonth()
-- This disconnects the connection point established
--
luacom.releaseConnection(calendar, event_handler, cookie)
-- This should NOT trigger the AfterUpdate event
--
calendar:NextMonth()
\end{verbatim}

Notice that \verb+addConnection+ also returns a cookie. A call to
\verb+releaseConnection+ needs both the event sink and the cookie to release the
connection. The old (pre-1.3) syntax of \verb+releaseConnection+ (ommiting the event sink
and cookie) still works but will only release the last connection made (but there will
not be leaks; all connections are released when the object is garbage-collected).

\paragraph{Message loop}
To receive events, it is necessary to have a message loop in the thread
that owns the object that is receiving the events. All events are dispatched
through a Windows message queue created
during \com\ initialization. Without a message loop, the event objects
implemented by \luacom\ will never
receive method calls from the \com\ objects they are registered with. Out-of-process
\com\ servers implemented with \luacom\ also need a message loop to be able to service
method calls (one is provided by calling \verb+luacom.DetectAutomation+).


\subsection{Parameter Passing}

  \luacom\ has some policies concerning parameter passing. They
  specify how \luacom\ will translate \com\ parameter lists to \lua\
  and vice-versa. There are two different situations to which these
  policies apply: calling a method of a \com\ object from \lua\ and
  calling a \lua\ function from \com. The main question here is how to
  deal with the different types of parameters supported by \com\
  (``in'' parameters, ``out'' parameters, ``in-out'' parameters,
  ``optional'' parameters and ``defaultvalue'' parameters). There is
  also a special policy concerning generic \luacom\ objects.

\subsubsection{Calling \com\ from \lua}

  This situation happens when accessing a property or calling a method
  of a \com\ object through the \luacom\ object. Here follows a
  sample:

\begin{verbatim}
word = luacom.GetObject("Word.Application")
-- Here we are calling the "Move" method of the Application object of
-- a running instance of Microsoft(R) Word(R)
word:Move(100,100)
\end{verbatim}

  In this situation, there are two steps in the parameter passing
  process:

\begin{enumerate}
\item convert \lua\ parameters to \com\ (this will be called the
``lua2com'' situation);
\item convert \com's return value \emph{and} output values back to
\lua\ (this will be called the ``com2lua'' situation).
\end{enumerate}

\paragraph{lua2com situation} The translation is done based on the
type information of the method (or property); it's done following the
order the parameters appear in the type information of the method. The
\lua\ parameters are used in the same order. For each parameter there
are three possibilities:

\begin{description}
\item [The parameter is an ``in'' parameter] \luacom\ gets the first
\lua\ parameter not yet converted and converts it to \com\ using
\luacom\ type conversion engine.

\item [The parameter is an ``out'' parameter] \luacom\ ignores this
parameter, as it will only be filled by the called method. That is,
the ``out'' parameters SHOULD NOT appear in the \lua\ parameter list.

\item [The parameter is an ``in-out'' parameter] \luacom\ does the
same as for ``in'' parameters.

\end{description}

When the caller of the method wants to omit a parameter, it must pass
the \ident{nil} value; \luacom\ then proceeds accordingly, informing
the called method about the omission of the parameter. If the
parameter has a default value, it is used instead. Notice that
\luacom\ does not complain when one omits non-optional parameters. In
fact, \luacom\ ignores the fact that a parameter is or isn't
optional. It leaves the responsibility for checking this to the
implementation of the called method.

\paragraph{com2lua situation} When the called method finishes,
\luacom\ translates the return value and the output values (that is,
the values of the ``out'' and ``in-out'' parameters) to \lua\ return
values. That is, the method return value is returned to the \lua\ code
as the first return value; the output values are returned in the order
they appear in the parameter list (notice that here we use the \lua\
feature of multiple return values). If the method does not have return
values, that is, is a ``\ident{void}'' method, the return values will
be the output values. If there are no output values either, then there
will be no return values.

The called method can omit the return value or the output values;
\luacom\ them will return \ident{nil} for each omitted value.

To illustrate these concepts, here follows a sample of these
situations. First, we show an excerpt of an \ident{ODL} file
describing a method of a \com\ object:

\begin{verbatim}
  HRESULT TestShort(
    [in] short p1, // an "in" parameter
    [out] short* p2, // an "out" parameter
    [in,out] short* p3, // an "in-out" parameter
    [out,retval] short* retval); // the return value
\end{verbatim}

  Now follows a sample of what happens when calling the method:

\begin{verbatim}
-- assume that "com" is a LuaCOM object
-- Here we set p1 = 1, p3 = 2 and leave p2 uninitialized
-- When the method returns, r1 = retval and r2 = p2 and r3 = p3
r1, r2, r3 = com:TestShort(1,2)
-- Below is WRONG! The are only two in/in-out parameters! Out parameters
-- are ignored in the lua2com parameter translation
r1, r2, r3 = com:TestShort(1,2,3) -- WRONG!
-- Here p1 = 1, p2 is uninitialized and p3 is omitted.
r1, r2, r3 = com:TestShort(1)
-- Here we ignore the output value p3
r1,r2 = com:TestShort(1)
-- Here we ignore all output values (including the return value)
com:TestShort(1,2)
\end{verbatim}

\paragraph{Generic \luacom\ objects}

When dealing with generic \luacom\ objects, the binding adopts a
different policy: all \ident{Lua} parameters are converted to COM ones
as ``in-out'' parameters. \luacom\ assumes that these methods always
return a value; if the called method does not return anything,
\luacom\ pushes a \ident{nil} value\footnote{This feature allows a
clear distinction between the return value and the in-out parameters,
as all parameters will end up being returned.}. As all parameters are
set as ``in-out'', all of them will be returned back to Lua, modified
or not by the called method.


\subsubsection{Calling \lua\ from \com}

  This situation happens when one implements a \com\ \dispinterface\
  in \lua. The \activex\ binding has to translate the \com\ method
  calls to \lua\ function calls. The policy here concerning parameter
  list translation is the same as the one above, just exchanging
  ``Lua'' for ``COM'' and vice-versa. That is, all ``in'' an
  ``in-out'' \com\ parameters are translated to parameters to the
  \lua\ function call (the output parameters are ignored). When the
  call finishes, the first return value is translated as the return
  value of the \com\ method and the other return values are translated
  as the ``in-out'' and ``out'' values, following the order they
  appear in the method's type information. Continuing the previous
  example, here we show the implementation of a method callable from
  \com:

\begin{verbatim}
implementation = {}
-- This method receives TWO in/in-out parameters
function implementation:TestShort(p1, p2)
  -- the first one is the retval, the second the first out param
  -- the third the second out param (in fact, an in-out param)
  return p1+p2, p1-p2, p1*p2
end
-- Implements an interface
obj = luacom.ImplInterface(implementation, "TEST.Test", ITest)
-- calls the function implementation:TestShort via COM
r1, r2, r3 = obj:TestShort(1,2)
\end{verbatim}



\subsection{Exception Handling}
When a run time error occurr when using \luacom's methods or objects,
there are two possible actions \luacom\ can take:

\begin{itemize}
\item to signal the error using \verb+lua_error+;
\item ignore the error, just doing nothing or returning some kind of
error value.
\end{itemize}

The run time errors can be divided into three types:

\begin{itemize}
\item errors inside API calls, like \ident{CreateObject};
\item errors when using \luacom\ objects (COM method calls);
\item errors inside COM objects implemented in Lua.
\end{itemize}

%TODO colocar referencia para secao do luacom_log
The third type of error is always translated into a COM exception
returned to the server. To ease debugging, these errors are also
logged (if the logging facility has been activated), as the server can
silenty ignore these exceptions, specially in events.

If the \luacom\ library is compiled with \ttext{VERBOSE} defined, then
a lot of informative messages are logged and all errors are displayed
within a dialog box. This helps debug errors inside events on the fly,
as these errors are commonly ignored by the server. Notice that this
options slows down \luacom\ and can generate very big log files.

The behaviour of \luacom\ for the other two types can be
customized. There is a table called \ident{config} inside the \luacom\
table. This table holds three fields related to error handling:

\begin{description}

\item[\ident{abort\_on\_API\_error}] if false, \luacom\ silently
fails on errors inside API calls. This is NOT true for errors caused
by supplying bad parameters: these always generate calls to
\verb+lua_error+. The default value for this field is \emph{false}.

\item[\ident{abort\_on\_error}] if false, errors inside method calls
and property accesses are also ignored, possibly return \verb+nil+
where a return value is expected. The default value for this field is
\emph{true}.

\item[\ident{last\_error}] every time a run time error occurr \luacom\
sets this field with the text describing the error. This field can be
used to check if some operation failed; just remember to set it to
\ident{nil} before the operation of interest.

\end{description}

\paragraph{Sample}

\begin{verbatim}
-- to make all LuaCOM errors runtime errors
luacom.config.abort_on_error = true
luacom.config.abort_on_API_error = true
-- to silently ignore all errors
luacom.config.abort_on_error = false
luacom.config.abort_on_API_error = false
-- catching an ignored error
luacom.config.last_error = nil
obj:RunMethod(x,y)
if luacom.config.last_error then
  print("Error!")
  exit(1)
end
\end{verbatim}

All errors are also logged. Notice that some of the logged exceptions
are not really errors: they are side-effects of the extensive use of
exception handling inside \luacom\ code.






\section{Type Conversion}
\label{sect:type_conv}

%TODO: passar tudo o que for comportamento diferente do VB ou que
%possa confundir para a se��o release information

\luacom\ is responsible for converting values from \com\ to \lua\ and
vice versa. Most of the types can be mapped from COM to Lua and vice
versa without trouble. But there are some types for which the mapping
is not obvious. \luacom\ then uses some predefined rules to do the
type conversion. These rules must be known to avoid misinterpretation
of the conversion results and to avoid errors.



\subsection{Boolean values}
\label{subsect:boolean}

\luacom\ uses the boolean values \verb+true+ and \verb+false+ and
does not work with the older Lua 4 convention (\verb+nil+ and
non-\verb+nil+;).

\subsection{Pointers to \idispatch\ and \luacom\ objects}

A pointer to \idispatch\ is converted to a \luacom\ object whose
implementation is provided by this pointer. If the object is implemented by local
Lua table, then the pointer is converted to this table. A \luacom\ object is
converted to \com\ simply passing its interface implementation to
\com.


\subsection{Pointers to \ident{IUnknown}}

\luacom\ just allows passing and receiving \ident{IUnknown} pointers;
it does not operate on them. They are converted from/to userdatas with a
specific metatable.

\subsection{Arrays and Tables}

If the table does not have a \verb+__tocom+ metamethod,
\luacom\ first checks if the
table can be describing a variant. A table is a variant if it has a
\verb+Type+ field. This field must have a string that tells how the
\verb+Value+ field of the table must be converted. Possible values
for \verb+Type+ are \verb+string+, \verb+bool+, \verb+error+,
\verb+null+, \verb+currency+, \verb+decimal+, \verb+double+,
\verb+float+, \verb+int8+, \verb+uint8+, \verb+int4+, \verb+uint4+,
\verb+int2+, \verb+uint2+, \verb+int1+, \verb+uint1+, \verb+int+, and
\verb+uint+. Each corresponds to a variant type.

If the table is not describing a variant, then it may be describing
a date. A table is a date if it has one of those fields: \verb+Day+,
\verb+DayOfWeek+, \verb+Month+, \verb+Year+, \verb+Hour+,
\verb+Minute+, \verb+Second+, \verb+Milliseconds+. \luacom\
initializes the date with the fields that are present; the others
are kept at their default values.

If the table is not a date, \luacom\ converts \lua\ tables to \ident{SAFEARRAY}'s and
vice-versa. To be converted, \lua\ tables must be ``array-like'', that
is, all of its elements must be or ``scalars'' or tables of the same
length. These tables must also be ``array-like''. Here are some
samples of how is this conversion done:

\begin{center}
\begin{tabular}{|c|c|} \hline

\lua\ table & Safe Array \\ \hline \hline

\verb+table = {"name", "phone"}+  &
\begin{minipage}{5cm}
\begin{displaymath}
\left[\begin{array}{cc}
  "name"     & "phone"   \\
\end{array}\right]
\end{displaymath}
\end{minipage}
  \\ \hline

\verb+table = {{1,2},{4,9}}+          &
\begin{minipage}{5cm}
\begin{displaymath}
\left[\begin{array}{cc}
        1  & 2   \\
         4 & 9   \\
\end{array} \right]
\end{displaymath}
\end{minipage}
      \\ \hline
\end{tabular}
\end{center}

If the table has the conversion metamethod, \luacom\ uses it
to guide the conversion. If the metamethod is a method, \luacom\
calls it, passing the table and the COM type. The method should return a COM object
that \luacom\ will pass on. If the metamethod is a table, \luacom\ will
look for a \verb+typelib+ field, an \verb+interface+ field, and a \verb+coclass+ field,
and pass those as arguments to the \verb+ImplInterfaceFromTypelib+ API call. If the table
does not have a \verb+typelib+ field, \luacom\ will look for a \verb+progid+ field and
an \verb+interface+ field, and pass those to the \verb+ImplInterface+ API call. Either way,
\luacom\ will pass the returned object to COM.

\subsection{\ident{CURRENCY} type}
The \ident{CURRENCY} values are converted to \lua\ as numbers. When
converting a value to \com\ where a \ident{CURRENCY} is expected,
\luacom\ accepts both numbers and strings formatted using the current
locale for currency values. Notice that this is highly dependent on
the configuration and \luacom\ just uses the VARIANT conversion
functions.

\subsection{\ident{DATE} type}

When converting from \com\ to \lua, the default behavior is to transform \ident{DATE}
values to strings formatted according to the current locale. The
converse is true: \luacom\ converts strings formatted according to the
current locale to \ident{DATE} values.

The script can change the conversion from strings to tables by
setting the \verb+DateFormat+ field of the \verb+luacom+ table (the
\luacom\ namespace) to the string \verb+"table"+. The table will
have \verb+Day+, \verb+DayOfWeek+, \verb+Month+, \verb+Year+,
\verb+Hour+, \verb+Minute+, \verb+Second+, and \verb+Milliseconds+
fields. To return the conversion to strings, set the
\verb+DateFormat+ field to \verb+"string"+. Be careful with this
feature, as it may break compatibility with other scripts.

\subsection{Variants}

When converting from \com\ to \lua, the default behavior is to
transform variant values to the closest Lua type. The script can
change the conversion from Lua types to a table describing the
variant, by setting the \verb+TableVariants+ field of the
\verb+luacom+ table(the \luacom\ namespace) to \verb+true+. The
tables will have a \verb+Type+ field telling the original type of
the variant, and a \verb+Value+ field containing the conversion to
the closest Lua type. Be careful with this feature, as it may break
compatibility with other scripts.

\subsection{Error Handling}
When \luacom\ cannot convert a value from or to \com\, it issues an
exception, which may be translated to a \verb+lua_error+ or to a \com\
exception, depending on who is the one being called.


\section{Other Objects}

\luacom\ deals with other objects besides COM Automation ones. Here we
describe them briefly.

\subsection{The Enumerator Object}
This object is a proxy for a COM object that implements the
\ident{IEnumVARIANT} interface. It translates the calls made to fields
of the table to method calls using that interface. Enumerators arise
often when dealing with collections. To obtain an enumerator for a
collection, use the Lua API method \ident{GetEnumerator}. Example:

\begin{verbatim}
--
-- Sample use of enumerators
--
-- Gets an instance
word = luacom.GetObject("Word.Application")
-- Gets an enumerator for the Documents collection
docs_enum = luacom.GetEnumerator(word.Documents)
-- Prints the names of all open documents
doc = docs_enum:Next()
while doc do
  print(doc.Name)
  doc = docs_enum:Next()
end
\end{verbatim}

The Extended Lua API method \ident{pairs} allows the traversal of the
enumeration using Lua's \ident{for} statement. The sample above can be
rewritten this way:

\begin{verbatim}
--
-- Sample use of enumerators
--
-- Gets an instance
word = luacom.GetObject("Word.Application")
-- Prints the names of all open documents
for index, doc in luacom.pairs(word.Documents) do
  print(doc.Name)
end
\end{verbatim}



\subsection{The Connection Point Container Object}

This object allows a COM object implemented using \luacom\ to send
events to its client. It's used primarily when implementing COM object
in Lua, so see chapter \ref{sec:impl-com-objects} for more
information.

\subsection{The Typelib and Typeinfo Objects}
These objects allow the navigation through the type descriptions of a
\luacom\ object or of a type library. They are proxies for the
interfaces \ident{ITypeLib} and \ident{ITypeInfo}, although not all
methods are available. For more information, see sections
\ref{sec:type-library-object} and \ref{sec:type-inform-object}.





\chapter{Implementing COM objects and controls in Lua}
\label{sec:impl-com-objects}

\section{Introduction}
With \luacom\ it is possible to implement full-fledged COM objects
and OLE controls using Lua. Here we understand a COM object as a
composite of these parts:

\begin{itemize}
\item a server, which implements one or more COM objects;
\item registry information, which associates a CLSID (Class ID) to
a triple (\emph{server, type library, default interface)};
\item a ProgID (Programmatic Identifier) which is a name associated to
a CLSID;
\item a type library containing a CoClass element.
\end{itemize}

The registry information maps a ProgID to a CLSID, which is, in
turn, mapped to a server. The type information describes the
component --- that is, which interfaces it exposes and what is the
default interface.

\luacom\ simplifies these tasks by providing some helper functions to
deal with registration and instantiation of COM servers. \luacom\ suports
both local (EXE) and in-process (DLL) servers.

\luacom\ also provides helper functions to register and instantiate
OLE controls (with their user interface embedded in the hosting
application). This kind of object needs an in-process server and a
supported Lua GUI toolkit (IUP, for now).

\section{Is it really useful?}
Some might argue that it would be better to implement COM object in
languages like C++ or Visual Basic\registered . That's true in many
situations, and false in several others. First, dealing with COM is
not easy and \luacom\ hides most its complexities; besides that, there
is another compelling reason for using \luacom\, at least in some
situations: the semantics of Lua tables and the way \luacom\ is
implemented allows one to do some neat things:

\begin{itemize}
\item to expose as a COM object any object that can be accessed via
Lua through a table. These might be CORBA objects, C++ objects, C
structures, Lua code, etc. Using this feature, a legacy application or
library may be ``upgraded'' to COM world with little extra work;

\item to use COM objects anywhere a Lua table is expected. For
example, a COM object might be ``exported'' as a CORBA object,
accessible through a network;

\item to add and to redefine methods of an instance of a COM
object. This might be very useful in the preceding situations: an
object of interest might be incremented and them exported to another
client.

\end{itemize}

Of course all this flexibility comes at some cost, primarily
performance. Anyway, depending on the application, the performance
drawback might be negligible.

\luacom\ does not solve all problems: there is still the need of a
type library, which must be build using third party tools.


\section{Terminology}

To avoid misunderstandings, here we'll supply the meaning we give to
some terms used in this chapter. We don't provide formal definitions:
we just want to ease the understanding of some concepts. To better
understand these concepts, see COM's documentation.

\begin{description}

\item [Component] a piece of software with some functionality that can
be used by other components. It's composed by a set of objects that
implement this functionality.

\item [Component Object] an object through which all the functionality
of a component can be accessed, including its other objects. This
object may have many interfaces.

\item [Application Object] A component object with a interface that
comprises all the top-level functionality of a component; the client
does not need to use other interfaces of the component object. This
concept simplifies the understanding of a component, as it puts all
its functionalities in an hierarchical manner (an application object
together with its sub-objects, which can only be accessed through
methods and properties of the application object).

\item [COM server] Some piece of code that implements one or more
component objects. A COM server must tell the other applications and
components which component objects it makes available. It does so
\emph{exposing} them.

\item[OLE control] An object that has an user interface, and can be
embedded inside other applications that have \emph{OLE containers}
(usually C++ or VB applications).

\item [CoClass] A type library describing a component should have a
CoClass entry, specifying some information about the component:

  \begin{itemize}

  \item a name, differentiating one CoClass from others in the same
  type library;

  \item its CLSID, the unique identifier that distinguishes this
  component from all others;

  \item the interfaces of the component object, telling which one is
  the default. In a typical situation, only one interface will be
  supplied; thus the component object could be called an Application
  object for that component;

  \item the source interface --- that is, the interface the component
  uses to send events to the client. This interface is not implemented
  by the component: it just \emph{uses} objects that implement this
  interface.

  \end{itemize}

\item [Lua Application Object] It's the Lua table used to implement
the Application Object.

\end{description}


\section{Building a \luacom\ COM server}

There are some steps to build a COM server using \luacom:

\begin{enumerate}
\item specify the component;
\item identify what is going to be exported: Lua application object
and its sub-objects;
\item build a type library for the component;
\item define the registration information for the component;
\item register the Component object;
\item implement and expose the COM objects;
\item add COM initialization and termination code.
\end{enumerate}


\subsection{Specify the component}

This is the first step: to define what functionality the component
will expose. This functionality is represented by an hierarchy of
objects, rooted in the Application object. Each of these objects
should implement an interface.

\paragraph{Example} Suppose we have a Lua library that implements the
access of databases contained in a specific DBMS. This library has
three types of objects: databases, queries and records. In COM world,
this could be represented by an Application object that opens
databases and returns a Database Object. A Database object has, among
others, a Query method. This method receives a SQL statement and
returns a Query object. The Query object is a collection, which can be
iterated using the parameterized property Records, which returns an
object of type Record.


\subsection{Objects to be exported}

The objects to be exported are those belonging to the hierarchy rooted
in the Application object. In Lua world, objects are ordinarily
represented as tables or userdatas. So it's necessary to identify (or
to implement) the Lua tables used to implement the objects to be
exported.



\subsection{Building the type library}

The type library should contain entries for all the interfaces of
exported objects and an entry for the CoClass, specifying the
interface of the Application object and the interface used to send
events.

The most common way to build a type library is to write an IDL
describing the type library and them use an IDL compiler, such as
Microsoft's\registered\ MIDL. Notice that all the interfaces must be
dispinterfaces --- that is, must inherit from \idispatch and must have
the flag \texttt{oleautomation}.



\subsection{Registration Information}

Here we must specify the information that is used by COM to locate the
component. See documentation of \verb+RegisterObject+.


\subsection{Registering the Component Object}

Before being accessed by other applications, the component object must
be registered in the system registry. This can be done with the \verb+RegisterObject+ API function.
This function receives a table of registration info for
the object. See the complete example for the fields of this table.



\subsection{Implementing and Exposing the Component}
%TODO: eventos

There are two different situations, which one demands different
actions:

\begin{description}
\item [Implementing the Application Object] Here we must use the
\luacom\ method \verb+NewObject+ to create a COM object and
bind it to the table of the Lua Application Object. Them this object must
be made available to other applications through \verb+ExposeObject+.

\item [Implementing other objects] The other objects of the component
are obtained via the Lua Application Object as return values of
functions or as values stored in the fields of the Lua Application
Object (that is, via property access). These object should be
implemented using \verb+ImplInterface+. They can be implemented
in the initialization (and then be stored somewhere) or can be
implemented on-demand (that is, each time a COM object should be
return, a call to \verb+ImplInterface+ is made).
\end{description}

Notice that the fields of the Lua table used to implement COM
component will only be accessible if they are present in the type
library. If not, they are invisible to COM.


\subsection{Initialization and Termination}

\subsubsection*{Initialization}
%TODO: eventos ( guardar objeto connection point container)

If you are implementing your own server, instead of using the builtin support, your server
must call the COM initialization functions (\texttt{OleInitialize} or \texttt{CoInitialize}) before \luacom\ is
started. Another initialization task is the implementation and
exposition of the COM objects. This task can be greatly simplified
using the C/C++ \luacom\ API function \verb+luacom_detectAutomation+.

If you want to use the builtin support, the only initialization necessary is to call
the \verb+DetectAutomation+ API function at the end of the script that implements your objects,
passing a table containing methods to register and expose your objects.

\subsubsection*{Termination}
The COM server must call (in Lua) \verb+RevokeObject+ for each
exposed object. Then it must call the COM termination functions AFTER
\verb+lua_close+ has been called; otherwise fatal errors may occur.



\section{Running the COM server}

A COM server built following the preceding guidelines can be used
as any other COM object --- that is, using \verb+CoCreateInstance+,
\verb+CreateObject+ or something like these.



\section{Generating Events}

The method \verb+NewObject+ returns a connection point container
object. This object allows the component to send events to its clients
just calling methods on this object, passing the expected
parameters. Return values are not allowed yet.

\section{Full Example}

This is an example of a Lua COM server. The example assumes this script is called testobj.lua:

\begin{verbatim}
-- This is the implementation of the COM object
path_to_obj = "\\Path\\To\\Script\\"

TestObj = {}

function TestObj:showWindow()
  print("Show!")
  events:OnShow()
end

function TestObj:hideWindow()
  print("Hide!")
  events:OnHide()
end

COM = {}

function COM:StartAutomation()
  -- creates the object using its default interface
  COMAppObject, events, e = luacom.NewObject(TestObj, "TEST.Test")
  -- This error will be caught by detectAutomation
  if COMAppObject == nil then
     error("NewObject failed: "..e)
  end
  -- Exposes the object
  cookie = luacom.ExposeObject(COMAppObject)
  if cookie == nil then
     error("ExposeObject failed!")
  end
end

function COM:Register()
  -- fills table with registration information
  local reginfo = {}
  reginfo.VersionIndependentProgID = "TEST.Test"
  reginfo.ProgID = reginfo.VersionIndependentProgID..".1"
  reginfo.TypeLib = "test.tlb"
  reginfo.CoClass = "Test"
  reginfo.ComponentName = "Test Component"
  reginfo.Arguments = "/Automation"
  reginfo.ScriptFile = path_to_script .. "testobj.lua"
  -- stores component information in the registry
  local res = luacom.RegisterObject(reginfo)
  if res == nil then
     error("RegisterObject failed!")
  end
end

function COM:UnRegister()
  -- fills table with registration information
  local reginfo = {}
  reginfo.VersionIndependentProgID = "TEST.Test"
  reginfo.ProgID = reginfo.VersionIndependentProgID..".1"
  reginfo.TypeLib = "test.tlb"
  reginfo.CoClass = "Test"
  -- removes component information from the registry
  local res = luacom.UnRegisterObject(reginfo)
  if res == nil then
     error("UnRegisterObject failed!")
  end
end

-- Starts automation server
return luacom.DetectAutomation(COM)
\end{verbatim}


\section{Building a Lua OLE control}

Most of what is needed to build an OLE control was already covered
in the last section. Controls are like ordinary \luacom\ objects,
but they are created by the \verb+NewControl+ API function instead
of \verb+NewObject+. The registration info table must also have
\verb+Control+ field set to \verb+true+.

The table that implements the control must also implement a few
additional methods, part of the control protocol. These are:

\begin{description}
\item[InitialSize] The control may use this method to return its
 initial size, in pixels.
\item[CreateWindow] Called when the control has to create its
window. The parameters to this function are the handle of the parent
window (an userdata), the initial position and initial size of the
window. The control must return an userdata with its window handle.
\item[SetExtent] Called whenever the host wants to change the size
of the control. The parameters are the new size. Must return
\verb+true+ if the control accepts the size change, and \verb+false+
otherwise.
\item[GetClass] Must return the class id of the control.
\item[DestroyWindow] Called when the host is finished with the control, and
it has to destroy its window and release its resources.
\end{description}

The \verb+demo/control+ directory of the \luacom\ distribution has
an example of a control.

%%%
%   Release Info Chapter
%%%

\chapter{Release Information}
\label{chap:release}

Here is provided miscellaneous information specific to the current
version of \luacom. Here are recorded the current limitations of
\luacom, its known bugs, the history of modifications since the former
version, technical details, etc.

\section{Limitations}
\label{sect:limitations}

Here are listed the current limitations of \luacom, as of the current
version, and information about future relaxation of this restrictions.

\begin{itemize}

\item \luacom\ currently supports only exposes COM objects as ``single
use'' objects. That might be circumvented by exposing many times the
same object. This restriction might be removed under request;

\item \luacom\ doesn't support \com\ methods with variable number of
parameters. This could be circumvented passing the optional parameters
inside a table, but this hasn't been tested. This may be implemented
under request;

\item \luacom\ doesn't provide access to \com\ interfaces that don't
inherit from the \idispatch\ interface. That is, only Automation objects
are supported. This restriction is due to the late-binding feature
provided by \luacom. It's possible to provide access to these \com\
interfaces via a "proxy" Automation object, which translate calls made
through automation to vtable (early-binding) calls. It's also possible
to implement this "proxy" directly using \luacom\ C/C++ API, but this
hasn't been tested nor tried.

\end{itemize}


\section{Known bugs}
\label{sect:bugs}
Here are recorded the known bugs present in \luacom. If any other bugs
are found, please report them through \luacom's home page.

\begin{itemize}

\item \luacom\ only implements late-bound interfaces but accepts a
QueryInterface for early-bound ones. This erroneous behavior is due
to the way a VB client sends events to the server. See subsection
\ref{sec:visu-basicc-issue};

\item when a table of \luacom\ objects (that is, a SAFEARRAY of
\idispatch\ pointers) is passed as a parameter to a COM object, these
\luacom\ objects might not be disposed automatically and may leak;

\item when a COM object implemented in Lua is called from VBScript,
the ``in-out'' parameters of type SAFEARRAY cannot be modified. If
they are, VBScript will complain with a COM error.

\end{itemize}




\section{Future Enhancements}
\label{sect:enhancements}

Besides the enhancements listed in the sections \ref{sect:limitations} and
\ref{sect:bugs}, there are other planned enhancements:

\begin{itemize}
\item to improve the overall performance of \luacom;
\item dynamic creation of type libraries;
\item better support for creating full-fledged \com\ objects using \lua.
\end{itemize}


\section{Important issues about \luacom}

\luacom\ is very similar to using other Automation-enabled languages
or environments (Visual Basic\registered, VBA, VBScript\texttrademark,
pycom etc). Nevertheless, there are some subtle differences that might
confuse the programmer: different syntax, unexpected behavior, etc. To
ease the task of the \luacom\ user, we grouped the information related
to these isses here.

\subsection{Problems instantiating COM objects}
Some COM objects can rest at in-process servers (implemented in DLL's)
and at local servers (implemented as separate processes). COM gives
preference to in-process servers, as they are faster. Nevertheless,
some applications may not work with \luacom\ when working as
in-process servers. One should instance the COM object supplying an
additional flag forcing the use of local servers. See documentation
for API function \ident{CreateObject}.

\subsection{Releasing COM objects from memory}
In a normal scenario, an out-of-process COM server should terminate
when all references to its objects are released. This may be
importart, as the creation of new instances might depend on the
absence of a running one. \luacom\ integrates the standard COM
mechanism of reference counting with Lua's garbage collection. This
works fine in most situations, but there are some situations which
demand a more careful analysis:

\begin{itemize}
\item to \emph{immediately} terminate the server process, it's
necessary to eliminate all references in Lua to the COM objects
residing in this process and then force a garbage-collection cycle;
\item sometimes a reference to a COM object may be stored by mistake
to a global variable and then forgotten. This may prevent the
server process to exit even when a method like ``Quit'' is called. To
avoid this problem, one might group all to references to a COM object
and its sub-objects in a single table to avoid ``lost'' references.
\end{itemize}

For more information, see section \ref{sec:object-disposal}.


\subsection{Receiving events}
When one wishes to receive events and notifications from a COM object,
a connection must be established using connection points. But that is
not enough: the client application must have a message loop running to
get these notifications. For more information, see section
\ref{sec:conn-points:-handl}.


\subsection{Extensible Interfaces}

Some objects that have type information describing their interface
(methods, properties, types of parameters etc), may add new methods
and properties at runtime. This means that these methods and
properties can only be accessed using the same mechanism \luacom\ uses
for Generic \com\ object. This have some implications:

\begin{itemize}
\item when accessing properties, it's mandatory to access them as
methods and to use the \ident{set} prefix to alter their values. If an
object \ident{foo} has a property \ident{color} not present in the
type information, it can only be accessed through \ident{foo:color()}
(read access) or \ident{foo:setcolor()} (write access);

\item when calling methods, all parameters are treated as in-out. This
means that, beyond the return value, a call to a method of this type
will return all the parameters back, whether or not modified by the
callee. Anyway, one can ignore these values simply not assigning them
to a variable, e.g. \ident{x = foo:method(a,b)} will ignore the values
of \ident{a} and \ident{b}, also returned by the call.

\end{itemize}

\com\ objects related to WMI have this behavior. For more information,
see section \ref{sec:extens-interf}.


\subsection{Visual Basic\registered\ issue}
\label{sec:visu-basicc-issue}
A COM server implemented with \luacom\ can be used in VB with no
trouble:

\begin{verbatim}
Public lc as Object
Set lc = CreateObject("MyCOMObject.InLuaCOM")
lc.showWindow
b = lc.getData(3)
lc.Quit
\end{verbatim}

But if one wants to received events generated by a COM object
implemented using \luacom, then it's necessary to use VB's
\verb+Public WithEvents+:

\begin{verbatim}
Public WithEvents obj as MyCOMObject.Application
Set obj = CreateObject("MyCOMObject.Application")
Private Sub obj_genericEvent()
  ' Put your event code here
End Sub
\end{verbatim}

Here there is a problem: when VB assigns the result of
\verb+CreateObject+ to \verb+obj+ variable, it tries to get an early
bound interface (as far as I know, VB only uses late-bound interfaces
with variables of type \verb+Object+). \luacom\ does not work with
early-bound interfaces (known as vtable). If you call any method using
the \verb+obj+ variable, VB will throw an exception.

The solution we adopted was to accept a QueryInterface for a
early-bound interface (thus allowing the use of \texttt{Public
WithEvents}). Then the client \emph{must} do a ``typecast'' to use
correctly the COM object:

\begin{verbatim}
Public WithEvents obj_dummy as MyCOMObject.Application
Public obj as Object
Set obj_dummy = CreateObject("MyCOMObject.Application")
Set obj = obj_dummy
\end{verbatim}

This way the client may call methods of the COM object using the
\texttt{obj} variable.






\section{History}

\subsection*{Version 1.4}

\begin{itemize}

\item Formally support Lua 5.1 (Lua 4.0/5.0 support dropped).

\item Makefile updated to MSVC++2005.  Visual Studio MSVC++2005 and VC6
  project files included (.sln/.vcproj + .dsp).

\item SafeArray Fixes:
  Fixed handling of empty SafeArrays and SafeArrays of one dimension;
  Fixed reversed order of indices in multidimensional SafeArray's;
  Added new safearrays demo.

\item Fixed handling BSTR's with embedded zeros. Bug428.

\item Support HtmlHelp in luacom.ShowHelp. Feature2010.

\item Renamed luacom-lua5-1.3.dll to luacom.dll.

\item Other bug fixes, such as with parameters, and code improvements
  (See Tracker on LuaForge project page.)

\end{itemize}

\subsection*{Version 1.3}

\begin{itemize}

\item Support for the new Lua package proposal (see http://www.keplerproject.org/compat)

\item OLE controls with embedded UI;

\item Representing variants with tables;

\item Conversion tag/metamethod for tables;

\item Representing dates with tables;

\item More than one event sink connected to an object;

\item A method of typelibs returned by GetTypeInfo exports
  all enumerations of the typelib to a table;

\item Removal of registered servers from registry (unregister);

\item Identifies when an interface pointer is in fact a local Lua table
  implementing a COM object;

\item Fixed memory leak with some out parameters;

\item Removed line break in some system exceptions.

\end{itemize}

\subsection*{Version 1.2}

\begin{itemize}

\item Can be loaded by Lua 5's \verb+require+ function;

\item In-process servers, fully implemented in Lua (no initialization code in C is necessary
for in-process servers, and for local servers using Lua 5);

\item Now 1-based arrays are correctly converted by \luacom;

\item UNICODE strings are correctly converted from/to ANSI ones by
LuaCOM;

\item byte arrays are now converted from/to strings with embedded zeros;

\item \luacom\ now has a limited support for loading and browsing type
information and type libraries. This includes the ability to import
type library constants (\ident{enum}'s) as Lua globals and the ability
to open the help information associated with a component;

\item objects implementing \ident{IEnumVARIANT} interface are now
supported. This means that collections can be used in \luacom\ in a
similar way as the are in VBScript\texttrademark;

\item implemented a log mechanism to simplify debugging;

\item \luacom\ now handles correctly COM calls with named
parameters\footnote{Notice that \luacom\ does not implement named
parameters; it just takes them when called from a COM client and
puts them.}. This caused problems when receiving Microsoft
Excel\registered\ events;

\item now it's possible to specify the context used to create an
instance of a COM object (whether it should be created as a local
server or as an in-process server);

\item non-ANSI code removed;

\item when faced with an \ident{IUnknown} pointer, \luacom\ now
queries it for \ident{IDispatch} or \ident{IEnumVARIANT} interfaces,
returning a \luacom\ object instead of an \ident{IUnknown} pointer;

\item improved error-handling: now \luacom\ allows the customization
of the actions to be taken when errors occur;

\item \luacom\ now supports the concept of default method: when one
uses a reference to a \luacom\ object as a function, \luacom\ does the
function call using the default method of that object;

\item part of the \lua API of \luacom\ now is implemented
in Lua 5. This eases the addition of new features and avoids
cramming the library. Nevertheless, this does not impact those who use
the binary release, as they carry the Lua code precompiled;

\item \ident{luacom.GetObject} now supports the use of monikers. Among
other thing, this makes possible to use WMI and to open document files
directly, e.g. \verb+luacom.GetObject("myfile.xls")+;

\item \ident{luacom.CreateObject} and \ident{luacom.GetObject} now
make an attemp to initialize the COM object via
IPersistStreamInit. Some objects refuse to work without this step.

\end{itemize}

\subsection*{Version 1.1}

\begin{itemize}
\item \luacom\ is now compatible with Lua 4 and Lua 5. It's just a
matter of linking with the right library;

\item when used with Lua 5, \luacom\ uses booleans to better match the
Automation types;

\item all functions of \luacom's Lua API are now grouped together in a
single table called \verb+luacom+, although they are still accessible
globally as \texttt{luacom\_<function>} in the Lua 4 version of the
library;

\item now it's possible to create instances of Microsoft\registered\
Office\registered\ applications (Excel\registered, PowerPoint\registered,
etc.). It was only possible to use them via GetObject; now you can
create a new instance of these applications using
\texttt{luacom.CreateObject};

\item when compiled with the \texttt{NDEBUG} flag, \luacom\ does not
use any kind of terminal output anymore (\texttt{printf},
\texttt{cout} etc). This could break some applications.

\end{itemize}



\subsection*{Version 1.0}

\begin{itemize}
\item property access modified: now parameterized properties must be
accessed as functions using a prefix to differentiate property read
and write. If the prefix is omitted, a property get is assumed;

\item syntax ``\verb+obj.Property(param)+'' is no longer supported. A
colon -- ``:'' -- must be used: ``\verb+obj:Property(param)+'';

\item better support for implementation of COM objects, including
registration and event generation;

\item Type conversion engine rewritten. Now it adheres more firmly to
the types specified in the type libraries;

\item binding rewritten to better support ``out'' and ``in-out''
parameters and to adhere more strictly to the recommended memory
allocation policies for COM;

\item COM objects without type information are now supported.
\end{itemize}

\subsection*{Version 0.9.2}

\begin{itemize}
\item removal of \ident{LUACOM\_TRUE} and \ident{LUACOM\_FALSE}
constants; now booleans follow the same convention of the C language;
\item memory and interface leaks fixed;
\item some functions of the API have slightly different names;
\item changes in memory allocation policy, to follow more strictly
practices recommended in \com\ documentation;
\item parameter passing policies changed;
\item added limited support for \ident{IUnknown} pointers;
\item changes in type conversion;
\item added limited support for implementing and registering \com\
objects in \lua
\end{itemize}


\subsection*{Version 0.9.1}
\begin{itemize}
\item conversion to Lua 4;
\item better handling of different kinds of type information (e.g. now
can access Microsoft Internet Explorer\registered\ object);
\item now handles more gracefully exceptions and errors;
\item added support for optional parameters with default values;
\item \luacom\ does not initializes \com\ libraries anymore; this is
left to the user;
\item more stringent behavior about the syntax of method calls and
property access (methods with ``:'' and properties with ``.'').
\end{itemize}




%%%
%   Reference section
%%%

\chapter{Reference}
\label{chap:reference}

%TODO: atualizar com novas funcoes

\section{The C/C++ API}

\subsection*{luacom\_open}
%TODO: documentar questao das threads
\subsubsection*{Prototype}

\begin{verbatim}
  void luacom_open(lua_State* L);
\end{verbatim}

\subsubsection*{Description}
This function initializes the \luacom\ library, creates the global
\verb+luacom+ table and fills it with \luacom\ methods in the given
Lua state. Notice that it's necessary to initialize \com\ before,
using \ident{OleInitialize} or \ident{CoInitialize} or something like
that.

\subsubsection*{Sample}

\begin{verbatim}

int main()
{
  lua_State *L = luaL_newstate();

  OleInitialize(NULL);

  luacom_open(L);

  .
  .
  .
}

\end{verbatim}




\subsection*{luacom\_close}

\subsubsection*{Prototype}

\begin{verbatim}
  void luacom_close(lua_State* L);
\end{verbatim}

\subsubsection*{Description}
This function is intended to clean up the data structures associated
with \luacom\ in a specific \lua\ state (\ident{L}). Currently, it does nothing,
but in future releases it could. So, do not remove it from your code!
It must be also called before the \com\ termination functions
(\ident{OleUninitialize} and \ident{CoInitialize}) and before
\ident{lua\_close}.


\subsubsection*{Sample}

\begin{verbatim}

int main()
{
  lua_State *L = luaL_newstate();

  OleInitialize(NULL);

  luacom_open(L);

  .
  .
  .

  luacom_close(L);

  lua_close(L);

  OleUninitialize();
}

\end{verbatim}




\subsection*{luacom\_detectAutomation}

\subsubsection*{Prototype}

\begin{verbatim}
  int luacom_detectAutomation(lua_State *L, int argc, char *argv[]);
\end{verbatim}

\subsubsection*{Description}
This function gets from the top of the Lua stack a table which should
hold two fields named ``StartAutomation'' and ``Register'' (these
fields should contain functions that implement these actions). Then it
searches the command line (provided \texttt{argc} and \texttt{argv})
for the switches ``/Automation'' or ``/Register''. If one of these
switches is found, it then calls the corresponding function in the Lua
table. Finally it returns a value telling what happened, so the caller
function may change its course of action (if needed).

This function is simply a helper for those implementing Automation
servers using \luacom. Most of the work should be done by the Lua
code, using the methods \verb+RegisterObject+,
\verb+NewObject+, and \verb+ExposeObject+.

\subsubsection*{Sample}

\begin{verbatim}
/*
 * com_object.cpp
 *
 * This sample C++ code initializes the libraries and
 * the COM engine to export a COM object implemented in Lua
 */

#include <ole2.h>

// libraries
extern "C"
{
#include <lua.h>
#include <lualib.h>
#include <lauxlib.h>
}
#include <luacom.h>

int main (int argc, char *argv[])
{
  int a = 0;

  CoInitialize(NULL);

  IupOpen();

  lua_State *L = luaL_newstate();

  luaL_openlibs(L);

  luacom_open(L);

  luaL_dofile(L, "implementation.lua");

  // Pushes the table containing the functions
  // responsible for the initialization of the
  // COM object

  lua_getglobal(L, "COM");

  // detects whether the program was invoked for Automation,
  // registration or none of that

  int result = luacom_detectAutomation(L, argc, argv);

  switch(result)
  {
  case LUACOM_AUTOMATION:
    // runs the message loop, as all the needed initialization
    // has already been performed
    MessageLoop();
    break;

  case LUACOM_NOAUTOMATION:
    // This only works as a COM server
    printf("Error. This is a COM server\n");
    break;

  case LUACOM_REGISTER:
    // Notifies that the COM object has been
    // registered
    printf("COM object successfully registered.");
    break;

  case LUACOM_AUTOMATION_ERROR:
    // detectAutomation found /Automation or /Register but
    // the initialization Lua functions returned some error
    printf("Error starting Automation");
    break;
  }

  luacom_close(L);
  lua_close(L);

  CoUninitialize();

  return 0;
}

-------
-- implementation.lua
--
--   This is a sample implementation of a COM server in Lua
--
require "luacom"

-- This is the implementation of the COM object
TestObj = {}

function TestObj:showWindow()
  dialog.show()
end

function TestObj:hideWindow()
  dialog.hide()
end

-- Here we create and populate the table to
-- be used with detectAutomation

COM = {}

-- This functions creates the COM object to be
-- exported and exposes it.
function COM:StartAutomation()
  -- creates the object using its default interface
  COMAppObject, events, e = luacom.NewObject(TestObj, "TESTE.Teste")
  -- This error will be caught by detectAutomation
  if COMAppObject == nil then
     error("NewObject failed: "..e)
  end
  -- Exposes the object
  cookie = luacom.ExposeObject(COMAppObject)
  if cookie == nil then
     error("ExposeObject failed!")
  end
end

function COM:Register()
  -- fills table with registration information
  local reginfo = {}
  reginfo.VersionIndependentProgID = "TESTE.Teste"
  reginfo.ProgID = reginfo.VersionIndependentProgID..".1"
  reginfo.TypeLib = "teste.tlb"
  reginfo.CoClass = "Teste"
  reginfo.ComponentName = "Test Component"
  reginfo.Arguments = "/Automation"
  -- stores component information in the registry
  local res = luacom.RegisterObject(reginfo)
  if res == nil then
     error("RegisterObject failed!")
  end
end

function COM:UnRegister()
  -- fills table with registration information
  local reginfo = {}
  reginfo.VersionIndependentProgID = "TESTE.Teste"
  reginfo.ProgID = reginfo.VersionIndependentProgID..".1"
  reginfo.TypeLib = "teste.tlb"
  reginfo.CoClass = "Teste"
  -- removes component information from the registry
  local res = luacom.UnRegisterObject(reginfo)
  if res == nil then
     error("UnRegisterObject failed!")
  end
end

\end{verbatim}


\subsection*{luacom\_IDispatch2LuaCOM}

\subsubsection*{Prototype}

\begin{verbatim}
  int luacom_IDispatch2LuaCOM(lua_State *L, void *pdisp_arg);
\end{verbatim}

\subsubsection*{Description}
This functions takes a pointer to \ident{IDispatch}, creates a
\luacom\ object for it and pushes it in the \lua\ stack. This function
is useful when one gets an interface for a \com\ object from
\ident{C/C++} code and wants to use it in \lua.


\subsubsection*{Sample}

\begin{verbatim}
void CreateAndExport(lua_State* L)
{
  // Creates the object
  IUnknown *obj = CreateObj();

  // Gets the IDispatch
  IDispatch* pdisp = NULL;
  QueryInterface(IID_IDISPATCH, &pdisp);

  // pushes onto lua stack
  luacom_IDispatch2LuaCOM(L, (void *) pdisp);
}
\end{verbatim}


\section{The Lua Standard API}

%%%% luacom.CreateObject
%TODO: atualizar explicacoes para parametros (ID pode ser
%progID ou CLSID,...

\subsection*{CreateObject}

\subsubsection*{Use}

\begin{verbatim}
luacom_obj = luacom.CreateObject(ID, creation_context, untyped)
\end{verbatim}

\subsubsection*{Description}
This method finds the Class ID referenced by the ID parameter
and creates an instance of the object with this Class ID. If there
is any problem (ProgID not found, error instantiating object), the
method returns nil.

\subsubsection*{Parameters}

\begin{tabular}{|c|c|}
\hline
\bf Parameter & \bf Type  \\ \hline \hline
ProgID & String \\ \hline
\end{tabular}

\subsubsection*{Return Values}

\begin{tabular}{|c|p{4cm}|}
\hline
\bf Return Item & \bf Possible Values \\ \hline \hline
luacom\_obj  & \luacom\ object\newline nil \\ \hline
\end{tabular}

\subsubsection*{Sample}

\begin{verbatim}
inet_obj = luacom.CreateObject("InetCtls.Inet")
if inet_obj == nil then
  print("Error! Object could not be created!")
end
\end{verbatim}



%%%% luacom.Connect

\subsection*{Connect}
%TODO: documentar necessidade de um loop de mensagens
\subsubsection*{Use}

  \begin{verbatim}
  implemented_obj, cookie = luacom.Connect(luacom_obj, implementation_table)
  \end{verbatim}

\subsubsection*{Description}
  This method finds the default source interface of the object
  \verb+luacom_obj+, creates an instance of this interface whose
  implementation is given by \verb+implementation_table+ and creates a
  connection point between the \verb+luacom_obj+ and the implemented
  source interface. Any calls made by the \verb+luacom_obj+ to the
  source interface implementation will be translated to Lua calls to
  member function present in the \verb+implementation_table+. If the
  method succeeds, the \luacom\ object implemented by
  \verb+implementation_table+, plus a cookie that identifies the connection, are returned; otherwise, \verb+nil+ is
  returned.

  Notice that, to receive events, it's necessary to have a Windows
  message loop.


\subsubsection*{Parameters}

\begin{tabular}{|c|c|}
\hline
\bf Parameter & \bf Type    \\ \hline \hline
\verb+luacom_obj+ & \luacom\ object     \\ \hline
\verb+implementation_table+ & Table or userdata\\ \hline
\end{tabular}

\subsubsection*{Return Values}

\begin{tabular}{|c|p{4cm}|}
\hline
\bf Return Item & \bf Possible Values \\ \hline \hline
implemented\_obj  & \luacom\ object\newline nil \\ \hline
cookie & number \\ \hline
\end{tabular}

\subsubsection*{Sample}

\begin{verbatim}
events_handler = {}
function events_handler:NewValue(new_value)
  print(new_value)
end
events_obj = luacom.Connect(luacom_obj, events_handler)
\end{verbatim}

\subsection*{ImplInterface}

\subsubsection*{Use}

  \begin{verbatim}
  implemented_obj = luacom.ImplInterface(impl_table, ProgID, interface_name)
  \end{verbatim}

\subsubsection*{Description}
This method finds the type library associated with the ProgID and
tries to find the type information of an interface called
``interface\_name''. If it does, then creates an object whose
implementation is ``impl\_table'', that is, any method call or property
access on this object is translated to calls or access on the members
of the table. Then it makes a \luacom\ object for the implemented
interface and returns it. If there are any problems in the process
(ProgID not found, interface not found, interface isn't a
\dispinterface), the method returns nil.

\subsubsection*{Parameters}

\begin{tabular}{|c|c|}
\hline
\bf Parameter & \bf Type    \\ \hline \hline
\verb+impl_table+ & table or userdata     \\ \hline
\verb+ProgID+ & string\\ \hline
\verb+interface_name+ & string\\ \hline
\end{tabular}

\subsubsection*{Return Values}

\begin{tabular}{|c|p{4cm}|}
\hline
\bf Return Item & \bf Possible Values \\ \hline \hline
implemented\_obj  & \luacom\ object\newline nil \\ \hline
\end{tabular}

\subsubsection*{Sample}

\begin{verbatim}
myobject = {}
function myobject:MyMethod()
  print("My method!")
end
myobject.Property = "teste"
luacom_obj = luacom.ImplInterface(myobject, "TEST.Test", "ITest")
-- these are done via Lua
myobject:MyMethod()
print(myobject.Property)
-- this call is done through COM
luacom_obj:MyMethod()
print(luacom_obj.Property)
\end{verbatim}


\subsection*{ImplInterfaceFromTypelib}

\subsubsection*{Use}

\begin{verbatim}
impl_obj = luacom.ImplInterfaceFromTypelib(
  impl_table,
  typelib_path,
  interface_name,
  coclass_name)
\end{verbatim}

\subsubsection*{Description}
This method loads the type library whose file path is
``typelib\_path'' and tries to find the type information of an
interface called ``interface\_name''. If it does, then creates an
object whose implementation is ``impl\_table'', that is, any method
call or property access on this object is translated to calls or
access on the members of the table. Then it makes a \luacom\ object
for the implemented interface and returns it. If there are any
problems in the process (ProgID not found, interface not found,
interface isn't a \dispinterface), the method returns nil. The
``coclass\_name'' parameter is optional; it is only needed if the
resulting \luacom\ object is to be passed to the methods
\verb+Connect+, \verb+AddConnection+ or
\verb+ExposeObject+. This parameter specifies the Component
Object class name to which the interface belongs, as one interface may
be used in more than one ``coclass''.


\subsubsection*{Parameters}

\begin{tabular}{|c|c|}
\hline
\bf Parameter & \bf Type    \\ \hline \hline
\verb+impl_table+ & table or userdata     \\ \hline
\verb+typelib_path+ & string\\ \hline
\verb+interface_name+ & string\\ \hline
\verb+coclass_name+ & (optional) string\\ \hline
\end{tabular}

\subsubsection*{Return Values}

\begin{tabular}{|c|p{4cm}|}
\hline
\bf Return Item & \bf Possible Values \\ \hline \hline
implemented\_obj  & \luacom\ object\newline nil \\ \hline
\end{tabular}

\subsubsection*{Sample}

\begin{verbatim}
myobject = {}
function myobject:MyMethod()
  print("My method!")
end
myobject.Property = "teste"
luacom_obj = luacom.ImplInterfaceFromTypelib(myobject, "test.tlb",
"ITest", "Test")
-- these are done via Lua
myobject:MyMethod()
print(myobject.Property)
-- this call is done through COM
luacom_obj:MyMethod()
print(luacom_obj.Property)
\end{verbatim}

\subsection*{GetObject}

\subsubsection*{Use}
%TODO: atualizar para moniker
\begin{verbatim}
luacom_obj = luacom.GetObject(ProgID)
luacom_obj = luacom.GetObject(moniker)
\end{verbatim}

\subsubsection*{Description}
The first version method finds the Class ID referenced by the ProgID parameter
and tries to find a running instance of the object having this Class
ID. If there is any problem (ProgID not found, object is not running),
the method returns nil.

The second version tries to find an object through its moniker. If there is
any problem, the method returns nil.

\subsubsection*{Parameters}

\begin{tabular}{|c|c|}
\hline
\bf Parameter & \bf Type  \\ \hline \hline
ProgID/moniker & String \\ \hline
\end{tabular}

\subsubsection*{Return Values}

\begin{tabular}{|c|p{4cm}|}
\hline
\bf Return Item & \bf Possible Values \\ \hline \hline
luacom\_obj  & \luacom\ object\newline nil \\ \hline
\end{tabular}

\subsubsection*{Sample}

\begin{verbatim}
excel = luacom.GetObject("Excel.Application")
if excel  == nil then
  print("Error! Could not get object!")
end
\end{verbatim}

\subsection*{NewObject/NewControl}

\subsubsection*{Use}

  \begin{verbatim}
  -- Creates a COM object
  implemented_obj, events_sink, errmsg = luacom.NewObject(impl_table, ProgID)
  -- Creates an OLE control
  implemented_obj, events_sink, errmsg = luacom.NewControl(impl_table, ProgID)
  \end{verbatim}

\subsubsection*{Description}
This method is analogous to \verb+ImplInterface+, doing
just a step further: it locates the default interface for the ProgID
and uses its type information. That is, this method creates a \lua\
implementation of a COM object's default interface. This is useful
when implementing a complete \com\ object in \lua. It also creates a
connection point for sending events to the client application and
returns it as the second return value. If there are any problems in
the process (ProgID not found, default interface is not a
\dispinterface\ etc), the method returns nil twice and returns the
error message as the third return value.

To send events to the client application, just call methods of the
event sink table returned. The method call will be translated to COM
calls to each connection. These calls may contain parameters (as
specified in the type information).


\subsubsection*{Parameters}

\begin{tabular}{|c|c|}
\hline
\bf Parameter & \bf Type    \\ \hline \hline
\verb+impl_table+ & table or userdata     \\ \hline
\verb+ProgID+ & string\\ \hline
\end{tabular}

\subsubsection*{Return Values}

\begin{tabular}{|c|p{4cm}|}
\hline
\bf Return Item & \bf Possible Values \\ \hline \hline
implemented\_obj  & \luacom\ object\newline nil \\ \hline
event\_sink       & event sink table\newline nil \\ \hline
errmsg            & error message in the case of failure\newline nil \\ \hline
\end{tabular}

\subsubsection*{Sample}

\begin{verbatim}
myobject = {}

function myobject:MyMethod()
  print("My method!")
end

myobject.Property = "teste"

obj, evt, err = luacom.NewObject(myobject, "TEST.Test")

-- these are done via Lua
myobject:MyMethod()
print(myobject.Property)
-- this call is done through COM
luacom_obj:MyMethod()
print(luacom_obj.Property)
-- here we sink events
evt:Event1()
\end{verbatim}

\subsection*{ExposeObject}

\subsubsection*{Use}

\begin{verbatim}
cookie = luacom.ExposeObject(luacom_obj)
\end{verbatim}

\subsubsection*{Description}
This method creates and registers a \emph{class factory} for
\ident{luacom\_obj}, so that other running applications can use it. It
returns a cookie that must be used to unregister the object. If the
method fails, it returns \ident{nil}.

ATTENTION: the object MUST be unregistered (using
\verb+RevokeObject+) before calling \ident{luacom\_close} or
\ident{lua\_close}, otherwise unhandled exceptions might occur.

\subsubsection*{Parameters}

\begin{tabular}{|c|c|}
\hline
\bf Parameter & \bf Type  \\ \hline \hline
luacom\_obj & \luacom\ object  \\ \hline
\end{tabular}

\subsubsection*{Return Values}

\begin{tabular}{|c|p{4cm}|}
\hline
\bf Return Item & \bf Possible Values \\ \hline \hline
cookie  & number\newline nil \\ \hline
\end{tabular}

\subsubsection*{Sample}

\begin{verbatim}
myobject = luacom.NewObject(impl_table, "Word.Application")
cookie = luacom.ExposeObject(myobject)
function end_of_application()
  luacom.RevokeObject(cookie)
end
\end{verbatim}


\subsection*{RegisterObject}

\subsubsection*{Use}

\begin{verbatim}
result = luacom.RegisterObject(registration_info)
\end{verbatim}

\subsubsection*{Description}
This method creates the necessary registry entries for a COM object,
using the information in \texttt{registration\_info} table. If the
component is successfully registered, the method returns a non-nil
value.

The \texttt{registration\_info} table must contain the following
fields\footnote{For a better description of these fields, see COM's
documentation.}:

\begin{description}
\item [VersionIndependentProgID] This field must contain a string
describing the programmatic identifier for the component,
e.g. ``MyCompany.MyApplication''.
\item [ProgID] The same as VersionIndependentProgID but with a version
number, e.g. ``MyCompany.MyApplication.2''.
\item [TypeLib] The file name of the type library describing the
component. This file name should contain a path, if the type library
isn't in the same folder of the executable. Samples:
\verb+mytypelib.tlb+, \verb+c:\app\test.tlb+, \verb+test.exe\1+ (this
last one can be used when the type library is bound to the executable
as a resource).
\item [Control] Must be \verb+true+ if the object is an OLE control,
and \verb+false+ or \verb+nil+ otherwise.
\item [CoClass] The name of the component class. There must be a
\texttt{coclass} entry in the type library with the same name or the
registration will fail.
\item [ComponentName] This is the human-readable name of the
component.
\item [Arguments] This field specifies what arguments will be supplied
to the component executable when started via COM. Normally it should
contain ``\texttt{/Automation}''.
\item [ScriptFile] This field specifies the full path of the script
  file that implements the component. Only used to register in-process
  servers.
\end{description}

This method is not a generic ``registering tool'' for COM
components, as it assumes the component to be registered is
implemented by the running executable during registration.

\subsubsection*{Parameters}

\begin{tabular}{|c|c|}
\hline
\bf Parameter & \bf Type  \\ \hline \hline
registration\_info & table with registration information  \\ \hline
\end{tabular}

\subsubsection*{Return Values}

\begin{tabular}{|c|p{4cm}|}
\hline
\bf Return Item & \bf Possible Values \\ \hline \hline
result  & \texttt{nil} or non-\texttt{nil} value\\ \hline
\end{tabular}

\subsubsection*{Sample}

\begin{verbatim}
-- Lua registration code
function RegisterComponent()
  reginfo.VersionIndependentProgID = "TESTE.Teste"
  -- Adds version information
  reginfo.ProgID = reginfo.VersionIndependentProgID..".1"
  reginfo.TypeLib = "teste.tlb"
  reginfo.CoClass = "Teste"
  reginfo.ComponentName = "Test Component"
  reginfo.Arguments = "/Automation"
  reginfo.ScriptFile = "teste.lua"
  local res = luacom.RegisterObject(reginfo)
  return res
end
\end{verbatim}

\subsection*{UnRegisterObject}

\subsubsection*{Use}

\begin{verbatim}
result = luacom.UnRegisterObject(registration_info)
\end{verbatim}

\subsubsection*{Description}
This method removes the registry entries for a COM object,
using the information in \texttt{registration\_info} table. If the
component is successfully unregistered, the method returns a non-nil
value.

The \texttt{registration\_info} table must contain the following
fields\footnote{For a better description of these fields, see COM's
documentation.}:

\begin{description}
\item [VersionIndependentProgID] This field must contain a string
describing the programmatic identifier for the component,
e.g. ``MyCompany.MyApplication''.
\item [ProgID] The same as VersionIndependentProgID but with a version
number, e.g. ``MyCompany.MyApplication.2''.
\item [TypeLib] The file name of the type library describing the
component. This file name should contain a path, if the type library
isn't in the same folder of the executable. Samples:
\verb+mytypelib.tlb+, \verb+c:\app\test.tlb+, \verb+test.exe\1+ (this
last one can be used when the type library is bound to the executable
as a resource).
\item [CoClass] The name of the component class. There must be a
\texttt{coclass} entry in the type library with the same name or the
registration will fail.
\end{description}

\subsubsection*{Parameters}

\begin{tabular}{|c|c|}
\hline
\bf Parameter & \bf Type  \\ \hline \hline
registration\_info & table with registration information  \\ \hline
\end{tabular}

\subsubsection*{Return Values}

\begin{tabular}{|c|p{4cm}|}
\hline
\bf Return Item & \bf Possible Values \\ \hline \hline
result  & \texttt{nil} or non-\texttt{nil} value\\ \hline
\end{tabular}

\subsubsection*{Sample}

\begin{verbatim}
-- Lua registration code
function UnRegisterComponent()
  reginfo.VersionIndependentProgID = "TESTE.Teste"
  -- Adds version information
  reginfo.ProgID = reginfo.VersionIndependentProgID..".1"
  reginfo.TypeLib = "teste.tlb"
  reginfo.CoClass = "Teste"
  local res = luacom.UnRegisterObject(reginfo)
  return res
end
\end{verbatim}

\subsection*{addConnection}

\subsubsection*{Use}

\begin{verbatim}
cookie = luacom.addConnection(client, server)
\end{verbatim}

\subsubsection*{Description}
This method connects two \luacom\ objects, setting the
\ident{server} as an event sink for the \ident{client} --- that is, the
client will call methods of the server to notify events (following the
\com\ model). This will only work if the \ident{client} supports
connection points of the \ident{server}'s type. If the method
succeeds, it returns the cookie that identifies the connection;
otherwise, it throws an error.

\subsubsection*{Parameters}

\begin{tabular}{|c|c|}
\hline
\bf Parameter & \bf Type  \\ \hline \hline
client & \luacom\ object  \\ \hline
server & \luacom\ object  \\ \hline
\end{tabular}

\subsubsection*{Return Values}

\begin{tabular}{|c|p{4cm}|}
\hline
\bf Return Item & \bf Possible Values \\ \hline \hline
cookie  & number \\ \hline
\end{tabular}

\subsubsection*{Sample}
\begin{verbatim}
obj = luacom.CreateObject("TEST.Test")
event_sink = {}
function event_sink:KeyPress(keynumber)
  print(keynumber)
end
event_obj = luacom.ImplInterface(
              event_sink, "TEST.Test", "ITestEvents")

cookie = luacom.addConnection(obj, event_obj)
\end{verbatim}

\subsection*{releaseConnection}

\subsubsection*{Use}

\begin{verbatim}
luacom.releaseConnection(client, event_sink, cookie)
\end{verbatim}

\subsubsection*{Description}
This method disconnects a \luacom\ object from an event sink.

\subsubsection*{Parameters}

\begin{tabular}{|c|c|}
\hline
\bf Parameter & \bf Type  \\ \hline \hline
client & \luacom\ object  \\ \hline
event\_sink & \luacom\ object  \\ \hline
cookie & \luacom\ object  \\ \hline
\end{tabular}

\subsubsection*{Return Values}
There are none.

\subsubsection*{Sample}
\begin{verbatim}
obj = luacom.CreateObject("TEST.Test")
event_sink = {}
function event_sink:KeyPress(keynumber)
  print(keynumber)
end
event_obj = luacom.ImplInterface(
              event_sink, "TEST.Test", "ITestEvents")
result = luacom.addConnection(obj, event_obj)
.
.
.
luacom.releaseConnection(obj)
\end{verbatim}

\subsection*{ProgIDfromCLSID}

\subsubsection*{Use}

\begin{verbatim}
progID = luacom.ProgIDfromCLSID(clsid)
\end{verbatim}

\subsubsection*{Description}
This method is a proxy for the Win32 function \ident{ProgIDFromCLSID}.

\subsubsection*{Parameters}

\begin{tabular}{|c|c|}
\hline
\bf Parameter & \bf Type  \\ \hline \hline
clsid & string \\ \hline
\end{tabular}

\subsubsection*{Return Values}
\begin{tabular}{|c|p{4cm}|}
\hline
\bf Return Item & \bf Possible Values \\ \hline \hline
progID  & string\newline nil \\ \hline
\end{tabular}


\subsubsection*{Sample}
\begin{verbatim}
progid = luacom.ProgIDfromCLSID("{8E27C92B-1264-101C-8A2F-040224009C02}")
obj = luacom.CreateObject(progid)
\end{verbatim}

\subsection*{CLSIDfromProgID}

\subsubsection*{Use}

\begin{verbatim}
clsid = luacom.CLSIDfromProgID(progID)
\end{verbatim}

\subsubsection*{Description}
It's the inverse of \verb+ProgIDfromCLSID+.


\subsection*{ShowHelp}

\subsubsection*{Use}

\begin{verbatim}
luacom.ShowHelp(luacom_obj)
\end{verbatim}

\subsubsection*{Description}
This method tries to locate the \ident{luacom\_obj}'s help file in its
type information and shows it.

\subsubsection*{Parameters}

\begin{tabular}{|c|c|}
\hline
\bf Parameter & \bf Type  \\ \hline \hline
luacom\_obj & \luacom\ object  \\ \hline
\end{tabular}

\subsubsection*{Return Values}
None.

\subsubsection*{Sample}
\begin{verbatim}
obj = luacom.CreateObject("TEST.Test")
luacom.ShowHelp(obj)
\end{verbatim}

\subsection*{GetIUnknown}

\subsubsection*{Use}

\begin{verbatim}
iunknown = luacom.GetIUnknown(luacom_obj)
\end{verbatim}

\subsubsection*{Description}
This method returns a userdata holding the \ident{IUnknown} interface
pointer to the \com\ object behind \ident{luacom\_obj}. It's important
to notice that \lua\ does not duplicates userdata: many calls to
\verb+GetIUnknown+ for the same \luacom\ object will return
the same userdata. This means that the reference count for the
\ident{IUnknown} interface will be incremented only once (that is, the
first time the userdata is pushed) and will be decremented only when
all the references to that userdata go out of scope (that is, when the
userdata suffers garbage collection).

One possible use for this method is to check whether two \luacom\
objects reference the same \com\ object.

\subsubsection*{Parameters}

\begin{tabular}{|c|c|}
\hline
\bf Parameter & \bf Type  \\ \hline \hline
luacom\_obj & \luacom\ object  \\ \hline
\end{tabular}

\subsubsection*{Return Values}

\begin{tabular}{|c|p{4cm}|}
\hline
\bf Return Item & \bf Possible Values \\ \hline \hline
iunknown  & userdata with IUnknown metatable\newline nil \\ \hline
\end{tabular}

\subsubsection*{Sample}

\begin{verbatim}
-- Creates two LuaCOM objects for the same COM object
-- (a running instance of Microsoft Word(R)  )
word1 = luacom.GetObject("Word.Application")
word2 = luacom.GetObject("Word.Application")
-- These two userdata should be the same
unk1 = luacom.GetIUnknown(word1)
unk2 = luacom.GetIUnknown(word2)
assert(unk1 == unk2)
\end{verbatim}

\subsection*{isMember}

\subsubsection*{Use}

\begin{verbatim}
answer = luacom.isMember(luacom_obj, member_name)
\end{verbatim}

\subsubsection*{Description}
This method returns true (that is, different from \ident{nil}) if
there exists a method or a property of the \ident{luacom\_obj} named
\ident{member\_name}.

\subsubsection*{Parameters}

\begin{tabular}{|c|c|}
\hline
\bf Parameter & \bf Type  \\ \hline \hline
luacom\_obj & \luacom\ object  \\ \hline
member\_name & string \\ \hline
\end{tabular}

\subsubsection*{Return Values}

\begin{tabular}{|c|p{4cm}|}
\hline
\bf Return Item & \bf Possible Values \\ \hline \hline
answer  & \ident{nil} or non-\ident{nil} \\ \hline
\end{tabular}

\subsubsection*{Sample}

\begin{verbatim}
obj = luacom.CreateObject("MyObject.Test")
if luacom.isMember(obj, "Test") then
  result = obj:Test()
end
\end{verbatim}

\subsection*{StartLog}

\subsubsection*{Use}

\begin{verbatim}
result = luacom.StartLog(log_file_name)
\end{verbatim}

\subsubsection*{Description}
This methods activates the log facility of \luacom, writing to the log
file all errors that occur. If the library was compiled with VERBOSE
defined, it also logs other informative messages like creation and
destruction of \luacom\ internal objects, method calls, etc. This can
help track down object leaks. The method returns \ttext{true} if the
log file could be opened, \ttext{false} otherwise.


\subsubsection*{Parameters}

\begin{tabular}{|c|c|}
\hline
\bf Parameter & \bf Type  \\ \hline \hline
log\_file\_name & string  \\ \hline
\end{tabular}

\subsubsection*{Return Values}

\begin{tabular}{|c|p{4cm}|}
\hline
\bf Return Item & \bf Possible Values \\ \hline \hline
result  & boolean  \\ \hline
\end{tabular}

\subsubsection*{Sample}

\begin{verbatim}
ok = luacom.StartLog("luacomlog.txt")
if not ok then
  print("log not opened")
end
\end{verbatim}

\subsection*{EndLog}

\subsubsection*{Use}

\begin{verbatim}
luacom.EndLog()
\end{verbatim}

\subsubsection*{Description}
This method stops the log facility (if it has been activated), closing
the log file.


\subsubsection*{Parameters}

None.

\subsubsection*{Return Values}

None.

\subsubsection*{Sample}

\begin{verbatim}
luacom.EndLog()
\end{verbatim}



\subsection*{GetEnumerator}

\subsubsection*{Use}

\begin{verbatim}
e = luacom.GetEnumerator(luacom_obj)
\end{verbatim}

\subsubsection*{Description}
This method returns a COM enumerator for a given \luacom\ object (if
it provides one). This is the same as calling the \ttext{\_NewEnum}
method, at least for the majority of the objects. The enumerator
object is described in section \ref{sec:enumerator-object}.

\subsubsection*{Parameters}
\begin{tabular}{|c|c|}
\hline
\bf Parameter & \bf Type  \\ \hline \hline
luacom\_obj & \luacom\ object  \\ \hline
\end{tabular}

\subsubsection*{Return Values}

\begin{tabular}{|c|p{4cm}|}
\hline
\bf Return Item & \bf Possible Values \\ \hline \hline
e  & enumerator object or nil \\ \hline
\end{tabular}


\subsubsection*{Sample}

\begin{verbatim}
-- Prints all sheets of an open Excel Application
excel = luacom.GetObject("Excel.Application")
e = luacom.GetEnumerator(excel.Sheets)
s = e:Next()
while s do
  print(s.Name)
  s = e:Next()
end
\end{verbatim}

\section{Lua Extended API}

\subsection*{pairs}

\subsection*{GetType}

\subsection*{CreateLocalObject}

\subsection*{CreateInprocObject}

\subsection*{LoadConstants}

\subsection*{FillTypeInfo}

\subsection*{FillTypeLib}

\section{Enumerator Object}
\label{sec:enumerator-object}

The enumerator object is a proxy for the interface
\ident{IEnumVARIANT}. It can be obtained using the API method
\ident{GetEnumerator}.

\subsection*{Methods}

\begin{description}

\item[Next] returns the next object in the enumeration or \ident{nil}
if the end has been reached.

\item[Skip] skips the next object, returning \ident{true} if succeeded
of \ident{false} if not.

\item[Reset] restarts the enumerator.

\item[Clone] returns a new enumerator in the same state.

\end{description}


\section{Type Library Object}
\label{sec:type-library-object}

The type library object is a proxy for the interface
\ident{ITypeLib}. It can be obtained using the API method
\ident{LoadTypeLibrary} or the type information object method
\ident{GetTypeLib}.

\subsection*{Methods}

\begin{description}

\item[GetDocumentation] returns a table containing the fields
\ident{name}, \ident{helpstring}, \ident{helpcontext} and
\ident{helpfile} for the type library.

\item[GetTypeInfoCount] returns the number of type descriptions
contained in the type library.\

\item[GetTypeInfo(n)] returns an type information object for the n-th
type description.

\item[ShowHelp] tries to launch the help file associated with the type
library (if any).

\end{description}


\section{Type Information Object}
\label{sec:type-inform-object}


The type information object is a proxy for the interface
\ident{ITypeInfo}. It can be obtained using the API method
\ident{GetTypeInfo} or the type library object method
\ident{GetTypeInfo}.

\subsection*{Methods}

\begin{description}

\item[GetTypeLib] returns the containing type library object.

\item[GetFuncDesc(n)] returns a table describing the n-th function of
the type description. This table contains the following fields:
\ident{memid} (dispatch identifier), \ident{invkind} (invoke kind),
\ident{Params} (number of parameters), \ident{ParamsOpt} (number of
optional parameters), \ident{description}, \ident{helpfile},
\ident{helpcontext}, \ident{name}. Besides that, it stores an
array-like table called \ident{parameters} describing each parameter
of the function, with these fields: \ident{name}, \ident{type}.


\item[GetVarDesc(n)] returns a table describing the n-th variable (or
constant) in the type description. This table contains the following
fields: \ident{name}, \ident{value} (for constants only).

\item[GetDocumentation] returns a table with documentation for the type
description, with the fields \ident{name}, \ident{helpstring},
\ident{helpcontext} and \ident{helpfile}.

\item[GetTypeAttr] returns a table containing the type attributes for
the type description. This table holds the following fields:
\ident{GUID}, \ident{typekind}, \ident{Funcs} (number of functions),
\ident{Vars} (number of variables or constants) and \ident{ImplTypes}.
There is also a \ident{flags} field, containing a table that describes
the flags for this type description. This table contains the following
boolean fields: \ident{control}, \ident{appobject}, \ident{dispatchable},
\ident{oleautomation}, \ident{cancreate}.

\item[GetImplType(n)] For type descriptions of COM classes, this
returns the type information object for the nth interface of the COM
class.

\item[GetImplTypeFlags(n)] For type descriptions of COM classes, this
returns a table containing the implementation flags for the n-th
interface belonging to the COM class. This table holds the following
boolean flags: \ident{default}, \ident{source}, \ident{restricted},
\ident{defaultvtable}.

\item[ExportEnumerations] returns a table with all the enumerations in this typelib.
The keys are the enumeration names, and each one of them is a table, keyed by the
enumeration values.
\end{description}


\chapter{Credits}
\luacom\ has been developed by Renato Cerqueira, Vinicius
Almendra and Fabio Mascarenhas. The project is sponsored by TeCGraf (Technology Group
on Computer Graphics).

\end{document}
